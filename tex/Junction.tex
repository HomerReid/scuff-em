\documentclass[letterpaper]{article}

/home/homer/work/scuff-em/tex/scufftex.tex

\graphicspath{{figures/}}

%------------------------------------------------------------
%------------------------------------------------------------
%- Special commands for this document -----------------------
%------------------------------------------------------------
%------------------------------------------------------------

%------------------------------------------------------------
%------------------------------------------------------------
%- Document header  -----------------------------------------
%------------------------------------------------------------
%------------------------------------------------------------
\title {Handling of material junctions in {\sc scuff-em}}
\author {Homer Reid}
\date {June 18, 2012}


%------------------------------------------------------------
%------------------------------------------------------------
%- Start of actual document
%------------------------------------------------------------
%------------------------------------------------------------

\begin{document}
\pagestyle{myheadings}
\markright{Homer Reid: Handling of Material Junctions in {\sc scuff-em}}
\maketitle

\tableofcontents

%%%%%%%%%%%%%%%%%%%%%%%%%%%%%%%%%%%%%%%%%%%%%%%%%%%%%%%%%%%%%%%%%%%%%%
%%%%%%%%%%%%%%%%%%%%%%%%%%%%%%%%%%%%%%%%%%%%%%%%%%%%%%%%%%%%%%%%%%%%%%
%%%%%%%%%%%%%%%%%%%%%%%%%%%%%%%%%%%%%%%%%%%%%%%%%%%%%%%%%%%%%%%%%%%%%%
\newpage
\section{Material Junctions}

This is distinct 

%%%%%%%%%%%%%%%%%%%%%%%%%%%%%%%%%%%%%%%%%%%%%%%%%%%%%%%%%%%%%%%%%%%%%%
%%%%%%%%%%%%%%%%%%%%%%%%%%%%%%%%%%%%%%%%%%%%%%%%%%%%%%%%%%%%%%%%%%%%%%
%%%%%%%%%%%%%%%%%%%%%%%%%%%%%%%%%%%%%%%%%%%%%%%%%%%%%%%%%%%%%%%%%%%%%%
\newpage
\section{Half-RWG Basis Functions}

The half-RWG (HRWG) basis function associated with 
an edge $\vb L_\alpha$ of a panel $\pan$ is
$$ \vb h_\alpha(\vb x) = 
   \begin{cases}
   \displaystyle{ \frac{l_\alpha}{2A_\alpha}(\vb x - \vb Q_\alpha)}, 
   \qquad & \vb x \in \pan 
   \\[5pt]
   0, \qquad & \text{otherwise}
   \end{cases}
$$
where $l_\alpha$ is the edge length, $\vb Q_\alpha$ is 
the vertex of $\pan$ opposite the edge, and $A_\alpha$ 
is the area of $\pan$.
Introducing the \textit{indicator function} $\chi_\pan(\vb x)$, 
which equals unity for $\vb x\in \pan$ and 0 otherwise, 
we can rewrite this in the form
$$ \vb h_\alpha(\vb x) = 
   \frac{l_\alpha}{2A_\alpha}(\vb x - \vb Q_\alpha)\chi_\pan(\vb x).
$$
The novelty of the HRWG basis function is that the charge density
associated with it has both a surface-charge portion 
(like the ordinary RWG basis function) and a new \textit{line-charge}
density on edge $\vb L_\alpha$. Suppose we have an HRWG function
$\vb h_\alpha$ populated with strength $k_\alpha$. (Recall $k_\alpha$
has units of current/length.) Then the surface and line charge
densities associated with this current distribution are 
\begin{align*}
  \sigma(\vb x)
 &=
 -\frac{lk_\alpha}{i\omega A}\chi_\pan(\vb x)
\\
\textcolor{red}{ \lambda(\vb x)}
 &=
\textcolor{red}{-\frac{lk_\alpha}{i\omega}\chi_{\vb L_\alpha}(\vb x)}
\end{align*}

%%%%%%%%%%%%%%%%%%%%%%%%%%%%%%%%%%%%%%%%%%%%%%%%%%%%%%%%%%%%%%%%%%%%%%
%%%%%%%%%%%%%%%%%%%%%%%%%%%%%%%%%%%%%%%%%%%%%%%%%%%%%%%%%%%%%%%%%%%%%%
%%%%%%%%%%%%%%%%%%%%%%%%%%%%%%%%%%%%%%%%%%%%%%%%%%%%%%%%%%%%%%%%%%%%%%
\newpage
\section{Matrix Elements between RWG and HRWG Basis Functions}

\begin{align*}
  \Big\langle \vb f_\alpha 
  \Big|       \vb G
  \Big|       \vb h_\beta 
  \Big\rangle
&= -\frac{1}{k^2}
   \sum \sigma_{\alpha}^{\pm}
   \int_{\pan_\alpha^\pm} d\vb x 
   \int_{\vb L_\beta} d\vb x^\prime
   \left\{ \Big[\nabla \cdot \vb f_\alpha(\vb x)\Big]
           G(\vb x, \vb x^\prime)
           \Big[\nabla \cdot \vb h_\alpha(\vb x^\prime)\Big]
   \right\}
\end{align*}

\subsection*{Interaction between $\vb L$ and $\pan$}

\textit{Case 1. $\vb L$ is an edge of $\pan$.}

This is the situation of 
Figure \ref{EdgePanelInteractionFigure}\textbf{(a)}.
We parameterize points in $\pan$ and $\vb L$ as 
$$ \vb x = \vb V_0 + u\vb L + v\vb B
   \qquad
   \vb x^\prime = \vb V_0 + u^\prime \vb L
$$
$$ \vb x - \vb x^\prime = (u-u^\prime)\vb L + \vb v\vb B$$

\begin{align*}
  \int_{\pan{P}}\int_{\vb L} (\nabla \cdot \vb f) (\nabla \cdot h) G 
&=
  2l_\alpha l_\beta^2 
  \int_0^1 du \, \int_0^u \, dv \int_0^1 dv^\prime \,
  \frac{e^{ik|\vb x - \vb x^\prime|}}{4\pi |\vb x- \vb x^\prime|}
\end{align*}

\begin{align*}
  \int_0^1 du \int_0^u dv \int_0^1 du^\prime f(u,v,u^\prime)
&=\quad 
  \int_0^1 du^\prime \int_0^{u^\prime} du \int_0^u dv 
  f(u,v,u^\prime)
\\
&\quad+ 
  \int_0^1 du^\prime \int_{u^\prime}^1 du \int_0^{u-u^\prime} dv 
  f(u,v,u^\prime)
\\
&\quad+ 
  \int_0^1 du^\prime \int_{u^\prime}^1 du \int_{u-u^\prime}^u dv 
  f(u,v,u^\prime)
\end{align*}

\textit{Case 2. $\vb L$ shares one point in common with $\pan$.}

This is the situation of 
Figure \ref{EdgePanelInteractionFigure}\textbf{(b)}.
We parameterize points in $\pan$ and $\vb L$ as 
$$ \vb x = \vb V_0 + u\vb A + v\vb B
   \qquad
   \vb x^\prime = \vb V_0 + u^\prime \vb L.
$$
$$ \vb x - \vb x^\prime = u\vb A + v\vb B + u^\prime \vb L$$

\begin{align*}
  \int_{\pan{P}}\int_{\vb L} (\nabla \cdot \vb f) (\nabla \cdot h) G 
&=
  2l_\alpha l_beta^2 
  \int_0^1 du \, \int_0^u \, dv \int_0^1 dv^\prime \,
  \frac{e^{ik|\vb x - \vb x^\prime|}}{4\pi |\vb x- \vb x^\prime|}
\end{align*}

%%%%%%%%%%%%%%%%%%%%%%%%%%%%%%%%%%%%%%%%%%%%%%%%%%%%%%%%%%%%%%%%%%%%%%
%%%%%%%%%%%%%%%%%%%%%%%%%%%%%%%%%%%%%%%%%%%%%%%%%%%%%%%%%%%%%%%%%%%%%%
%%%%%%%%%%%%%%%%%%%%%%%%%%%%%%%%%%%%%%%%%%%%%%%%%%%%%%%%%%%%%%%%%%%%%%
\appendix

\newpage
\section{Taylor-Duffy transformation of integrals}

In this section we justify equation 
(\ref{TaylorDuffy}).

The integration region for the integral on the LHS of 
(\ref{TaylorDuffy}) is the direct product of a unit
right triangle in the $(u,v)$ plane with the unit 
interval on the $u^\prime$ axis, i.e. 
$$\text{region of integration}=\mathcal{R}=
  \begin{cases}
  0\le u \le 1 \\
  0\le v \le u \\
  0\le u^\prime \le 1.
  \end{cases}
$$

\subsection*{Step 1. Decompose region of integration into tetrahedra}

Following the spirit of the Taylor-Duffy method, we 
decompose $\mathcal{R}$ as the nonintersecting union 
of three tetrahedral regions:
$$ \mathcal{R} = \mathcal{R}_1 \cup
                 \mathcal{R}_2 \cup
                 \mathcal{R}_3
$$
where 
\begin{align*}
  \mathcal{R}_1 = 
  \begin{cases}
  0 \le u^\prime \le 1 \\
  0 \le u \le u^\prime \\
  0 \le v \le u
  \end{cases},
\qquad
  \mathcal{R}_2 = 
  \begin{cases}
  0 \le u \le 1 \\
  0 \le u^\prime \le u \\
  0 \le v \le u-u^\prime
  \end{cases}
\qquad
  \mathcal{R}_3 = 
  \begin{cases}
  0 \le u \le 1 \\
  0 \le u^\prime \le u \\
  u-u^\prime \le v \le u.
  \end{cases}
\end{align*}

\subsection*{Step 2. Introduce Duffy-transformed variables} 

The next step is to introduce Duffy-transformed variables for each 
of the three integration regions. The idea of the Duffy transform
is to single out one of the three variables as the ``independent''
variable (call it $w$), and then to ``measure the other two variables
in units of $w$.'' This ensures that the transformed versions of 
all three of the original variables $u,v,u^\prime$ are now expressed
as multiples of $w$, whereupon a factor of $w$ may be pulled out 
of radicals like $\sqrt{Au^2 + Bv^2 + Cu^{\prime 2}}$. When such radicals
appear in the denominator of the integrand, we then get overall 
multiplicative factors of $w$ in the denominator, which cancel with
the Jacobian of the variable transformation to yield 

It is also convenient, although not strictly necessary, to define our
Duffy-transformed variables $w,x,y$ in such a way that the region of
integration for all three of the transformed integrals is the unit
three-dimensional cube (i.e. $w,x,y$ all running from 0 to 1.) This 
allows the three integrals to be combined into a \textit{single} integral 
over the unit cube.

Such a variable transformation is 

%====================================================================%
\begin{align*}
\mathcal{R}_1:\qquad
    &u^\prime=w, \qquad u=wx, \qquad v=wxy, \quad \qquad \mathcal{J}=w^2 x
\\
\mathcal{R}_2: \qquad 
    &u=w, \qquad u^\prime=wx, \qquad v=w(1-x)y, \qquad \mathcal{J}=w^2 (1-x)
\\
\mathcal{R}_3: \qquad 
    &u=w, \qquad u^\prime=wx, \qquad v=w-wxy, \qquad \mathcal{J}= w^2x
\end{align*}
%====================================================================%

\subsection*{Step 3. Recast original integral as an integral over 
                     the unit cube}

%====================================================================%
\begin{align}
\int_0^1 du \int_0^u dv \int_0^1 du^\prime f(u,v,u^\prime)
\nn
=\int_0^1 dw \int_0^1 dx \int_0^1 dy 
  \bigg\{\quad & w^2 x f\Big(wx, wxy, w\Big)
\nn
       + &w^2(1-x) f\Big(w, w(1-x)y, wx\Big)
\nn
       + &w^2x  f\Big(w, w(1-xy), wx\Big)
  \bigg\}
\label{TransformedEdgePanelIntegral1}
\end{align}
%====================================================================%

\subsection*{Step 4. Evaluate the $w$ integral}

The integrand in question here is 
\begin{align*}
 f(u,v,u^\prime)
&=\frac{e^{ikR(u,v,u^\prime)}}{4\pi R(u,v,u^\prime)},
\qquad 
 R(u,v,u^\prime)=\Big|u\vb A + v\vb B - \vb u^\prime \vb L\Big|
\end{align*}
(where $\vb A=\vb L$ for the shared-edge case).
Applying (\ref{TransformedEdgePanelIntegral1}), we have
%====================================================================%
\begin{align}
&\int_0^1 du \int_0^u dv \int_0^1 du^\prime f(u,v,u^\prime)
\nn
&=\frac{1}{4\pi}\int_0^1 dx \int_0^1 dy \int_0^1 dw
\Big[  \frac{ w x     e^{ikwR_1(x,y)}}{R_1(x,y)} 
     + \frac{ w (1-x) e^{ikwR_2(x,y)}}{R_2(x,y)} 
     + \frac{ w x     e^{ikwR_3(x,y)}}{R_3(x,y)}
\Big] 
\label{TransformedEdgePanelIntegral2}
\end{align}
$$ R_1(x,y)=\Big| x\vb A + xy \vb B + \vb L|\Big|,
   \qquad
   R_2(x,y)=\Big| \vb A + (1-x)y \vb B + x\vb L|\Big|
   \qquad
   R_3(x,y)=\Big| \vb A + (1-xy) \vb B + x\vb L|\Big|.
$$
%====================================================================%
The $w$ integral may be evaluated in closed form:
\numeq{WIntegral}
{\int_0^1 w e^{wX} dw  = \frac{1}{2}e^X \cdot \text{ExpRel}(2,-X)}
where $\text{ExpRel}(N,X)$ is 
$e^X$, minus the first $N$ terms in its series expansion,
normalized so that $\text{ExpRel(N,0)}=1$:
\begin{align*}
 \text{ExpRel}(N,X) &\equiv
   \frac{N!}{X^N}\Big[e^{X} - 1 - X - \frac{X^2}{2!} - \cdots 
                             - \frac{X^{N-1}}{(N-1)!} \Big]
\\
&=
   \frac{N!}{X^N}\Big[\frac{X^N}{N!} + \frac{X^{N+1}}{(N+1)^1} + \cdots\Big] 
\\
&=
   1 + \frac{X}{N+1} + \frac{X^2}{(N+1)(N+2)} + \cdots
\end{align*}

Using (\ref{WIntegral}), equation (\ref{TransformedEdgePanelIntegral2})
becomes 
\begin{align*}
=\frac{1}{8\pi}\int_0^1 dx \int_0^1 dy \bigg\{
 \quad &\frac{x e^ikR_1}{R_1} e^{ikR_1} \cdot \text{ExpRel}(2,-ikR_1)
\\
      +&\frac{(1-x)}{R_2} e^{ikR_2} \cdot \text{ExpRel}(2,-ikR_2)
\\
      +&\frac{x}{R_3}e^{ikR_3} \cdot \text{ExpRel}(2,-ikR_3)
  \bigg\}
\end{align*}
where $R_1, R_2, R_3$ are functions of $x$ and $y$. This is a nonsingular
integral over the unit square which may be evaluated by straighforward
numerical cubature.

\end{document}
