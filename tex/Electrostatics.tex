\documentclass[letterpaper]{article}

/home/homer/work/scuff-em/tex/scufftex.tex

\graphicspath{{figures/}}

%------------------------------------------------------------
%------------------------------------------------------------
%- Special commands for this document -----------------------
%------------------------------------------------------------
%------------------------------------------------------------

%------------------------------------------------------------
%------------------------------------------------------------
%- Document header  -----------------------------------------
%------------------------------------------------------------
%------------------------------------------------------------
\title {Pure Electrostatics in \ls}
\author {Homer Reid}
\date {May 17, 2013}


%------------------------------------------------------------
%------------------------------------------------------------
%- Start of actual document
%------------------------------------------------------------
%------------------------------------------------------------

\begin{document}
\pagestyle{myheadings}
\markright{Homer Reid: Electrostatics in \ls}
\maketitle

\tableofcontents

%%%%%%%%%%%%%%%%%%%%%%%%%%%%%%%%%%%%%%%%%%%%%%%%%%%%%%%%%%%%%%%%%%%%%%
%%%%%%%%%%%%%%%%%%%%%%%%%%%%%%%%%%%%%%%%%%%%%%%%%%%%%%%%%%%%%%%%%%%%%%
%%%%%%%%%%%%%%%%%%%%%%%%%%%%%%%%%%%%%%%%%%%%%%%%%%%%%%%%%%%%%%%%%%%%%%
\section{Theory}

As in the usual \ls, we imagine geometries to consist
of homogeneous regions $\{\mc R_r\}$. Homogeneous region 
$\{\mc R_r\}$ has relative dielectric permittivity 
$\epsilon_r.$ 

The boundary of $\mc R_r$ is denoted $\partial \mc R_r$; 
it may consist of a single closed surface or a union of
surfaces, each of which may be individually open.
Thus we write 
%====================================================================%
\numeq{RrSs} { \partial \mc R_r = \cup \, \mathcal{S}_s }
%====================================================================%
Each surface $\mc S_s$ bounds precisely two regions; 
thus $\mc S_s$ appears on the RHS of equation (\ref{RrSs})
for two different values of $r$. 

On each surface lives a surface charge density 
$\sigma(\vb x)$.

Surface $\mc S_s$ may optionally be PEC, in which case it
must be assigned a fixed potential $V_s$ before the problem
can be solved. Physically, a PEC surface corresponds to
an infinitesimally thin conducting layer at the interface
between two dielectric regions, and the surface charge
on such a surface represents free charges supplied to the
surface by whatever batteries initially charged them up to
their specified  potentials $V_s.$

On the other hand, a non-PEC surface simply desribes the 
interface between two dielectric regions, and the 
surface charge on such a surface represents the divergence
of the bound volume polarization density in the dielectric  
region. Among other things, this means that the total
surface charge integrated over the boundary of a dielectric
region must vanish. There is no such constraint on 
the total surface charge on PEC surfaces.

%====================================================================%
%====================================================================%
%====================================================================%
\subsection*{Potentials and fields from charge densities: 
             continuous forms} 

Let $\vb x$ be a point in a region $\mc R_r$. The electrostatic potential 
and field at $\vb x$ are obtained by summing contributions from
surface charges on all surfaces bounding $\mc R_r$: 
%====================================================================%
\begin{align*}
\phi(\vb x) 
&= \frac{1}{4\pi\epsilon_0} 
   \oint_{\partial \mc R_r} 
   \frac{1}{|\vb x - \vb x^\prime|}
           \,\sigma(\vb x^\prime) \, d\vb x^\prime
\\
%--------------------------------------------------------------------%
\vb E(\vb x) 
&= \frac{1}{4\pi\epsilon_0}
   \oint_{\partial \mc R_r} 
   \frac{(\vb x-\vb x^\prime)}
        {|\vb x - \vb x^\prime|^3}
        \,\sigma(\vb x^\prime)\,d\vb x^\prime
\end{align*}
%====================================================================%
where the sum is over all panels $n$ that bound the region $r$.

%====================================================================%
%====================================================================%
%====================================================================%
\subsection*{Surface charge density expansion}

Now imagine approximating all surfaces as unions 
of triangular panels, 
$$ \mc S_{s} = \cup \, \mc P_{sn}$$
To panel $\mc P_{sn}$ we assign a scalar-valued 
``pulse'' basis function $b_{sn}(\vb x)$ that is 
1 on the panel and 0 elsewhere:
$$ b_{sn}(\vb x) = 
   \begin{cases}
    1, \qquad &\vb x \in \mc P_{sn} \\
    0, \qquad &\text{otherwise}.
  \end{cases}
$$
We approximate the surface charge density on $\mc S_s$
as an expansion in the $b_{sn}$ functions:
$$ \sigma(\vb x) \approx \sum_n \sigma_{sn} b_{sn}(\vb x)
   \qquad 
   \text{for }\vb x\in \mc S_s.
$$

%%%%%%%%%%%%%%%%%%%%%%%%%%%%%%%%%%%%%%%%%%%%%%%%%%%%%%%%%%%%%%%%%%%%%%
%%%%%%%%%%%%%%%%%%%%%%%%%%%%%%%%%%%%%%%%%%%%%%%%%%%%%%%%%%%%%%%%%%%%%%
%%%%%%%%%%%%%%%%%%%%%%%%%%%%%%%%%%%%%%%%%%%%%%%%%%%%%%%%%%%%%%%%%%%%%%
\subsection*{Potentials and fields from charge densities: 
             discretized forms} 

The electrostatic potential and field at $\vb x$ are 
%====================================================================%
\begin{align*}
\phi(\vb x) 
&= \frac{1}{4\pi\epsilon_0} \sum_{n} \sigma_n 
   \int_{\mc P_n} \frac{d\vb x^\prime}{|\vb x - \vb x^\prime|}
\\
%--------------------------------------------------------------------%
\vb E(\vb x) 
&= \frac{1}{4\pi\epsilon_0} \sum_{n} \sigma_n 
   \int_{\mc P_n} \frac{(\vb x-\vb x^\prime)d\vb x^\prime}
                       {|\vb x - \vb x^\prime|^3}
\end{align*}
where the sum is over all panels $n$ that bound the region $r$.
%====================================================================%

%%%%%%%%%%%%%%%%%%%%%%%%%%%%%%%%%%%%%%%%%%%%%%%%%%%%%%%%%%%%%%%%%%%%%%
%%%%%%%%%%%%%%%%%%%%%%%%%%%%%%%%%%%%%%%%%%%%%%%%%%%%%%%%%%%%%%%%%%%%%%
%%%%%%%%%%%%%%%%%%%%%%%%%%%%%%%%%%%%%%%%%%%%%%%%%%%%%%%%%%%%%%%%%%%%%%
\subsection*{Conditions on potentials and fields}

At PEC panels we impose the condition that the electrostatic
potential equal the specified potential for that conductor:
%====================================================================%
\begin{align*}
 \phi(\vb x) &= V_s, \qquad \text{for } \vb x \in \mc S_s.
\intertext{Galerkin-testing with the expansion functions for 
           surface $\mc S_s$, we find }
 \frac{1}{A_{sm}}
 \int_{\mc P_{sm}} \phi(\vb x) d\vb x &= V_s,
 \qquad \text{for all panels $\mc P_{sm}$ on surface }\mc S_m.
\end{align*}
Here $A_{sm}$ is the area of $\mc P_{sm}$.
%====================================================================%
At non-PEC panels we impose the condition that the normal 
electric field experience the requisite discontinuity. 
If $\vb x$ is a point on a surface $\mc S_s$ lying between
regions $\mc R_a$ and $\mc R_b$, 
\numeq{EFieldDiscontinuity}
{
 \epsilon_a \left|\pard{\phi}{\vbhat{n}}\right|_{\vb x^+}
=
 \epsilon_b \left|\pard{\phi}{\vbhat{n}}\right|_{\vb x^-}
}
where $\vbhat{n}$ is the surface normal pointing 
away from $\mc R_b$ into $\mc R_a$, and 
where $x^+$ and $x^-$ are points lying
infinitesimally displaced from $\vb x$ along
$\vbhat{n}$ into $\mc R_a$ and $\mc R_b$.

When we seek to enforce condition (\ref{EFieldDiscontinuity}) 
at a point $\vb x$ lying within a panel $\mc P_{sm}$
on $\mc S_s$, we find the following dichotomy:
%====================================================================%
\begin{enumerate}
  \item Surface charges on $\mc P_{sm}$ contribute to 
        the two sides of (\ref{EFieldDiscontinuity})
        with \textit{opposite} signs.
  \item Surface charges on all other panels, as well as the 
        external field soures, contribute
        to the two sides of (\ref{EFieldDiscontinuity})
        with the \textit{same} sign.
\end{enumerate}
%====================================================================%
\begin{align*}
&\epsilon_a 
   \left[ \frac{\sigma_m}{2} 
          + \sum_{n\ne m} \sigma_n \int_{\mc P_{n}}
            \frac{ \vbhat{n}_m \cdot (\vb x-\vb x^\prime)}
                 { |\vb x-\vb x^\prime|^3 } \, d\vb x^\prime
          + \vbhat{n}_m \cdot \vb E\sups{ext}(\vb x)
   \right]
\\
&\qquad
=\epsilon_b 
   \left[  -\frac{\sigma_m}{2} 
          + \sum_{n\ne m} \sigma_n \int_{\mc P_{n}}
            \frac{ \vbhat{n}_m \cdot (\vb x-\vb x^\prime)}
                 { |\vb x-\vb x^\prime|^3 } \, d\vb x^\prime
          + \vbhat{n}_m \cdot \vb E\sups{ext}(\vb x)
   \right]
\end{align*}
or 
$$ \sigma_m 
  + \Delta_{ab}
    \sum_{n\ne m} \sigma_n \int_{\mc P_{n}}
    \frac{ \vbhat{n}_m \cdot (\vb x-\vb x^\prime)}
         { |\vb x-\vb x^\prime|^3 } \, d\vb x^\prime
   = -\Delta_{ab}\vbhat{n}_m \cdot \vb E\sups{ext}(\vb x)
$$ 
$$\Delta_{ab} = \frac{\epsilon_a-\epsilon_b}{\epsilon_a + \epsilon_b}.$$

%%%%%%%%%%%%%%%%%%%%%%%%%%%%%%%%%%%%%%%%%%%%%%%%%%%%%%%%%%%%%%%%%%%%%%
%%%%%%%%%%%%%%%%%%%%%%%%%%%%%%%%%%%%%%%%%%%%%%%%%%%%%%%%%%%%%%%%%%%%%%
%%%%%%%%%%%%%%%%%%%%%%%%%%%%%%%%%%%%%%%%%%%%%%%%%%%%%%%%%%%%%%%%%%%%%%


\end{document}
