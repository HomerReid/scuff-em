\documentclass[letterpaper]{article}

/home/homer/work/scuff-em/tex/scufftex.tex

\graphicspath{{figures/}}

%------------------------------------------------------------
%------------------------------------------------------------
%- Special commands for this document -----------------------
%------------------------------------------------------------
%------------------------------------------------------------

%------------------------------------------------------------
%------------------------------------------------------------
%- Document header  -----------------------------------------
%------------------------------------------------------------
%------------------------------------------------------------
\title {Pure Electrostatics in \ls}
\author {Homer Reid}
\date {May 17, 2013}


%------------------------------------------------------------
%------------------------------------------------------------
%- Start of actual document
%------------------------------------------------------------
%------------------------------------------------------------

\begin{document}
\pagestyle{myheadings}
\markright{Homer Reid: Electrostatics in \ls}
\maketitle

\tableofcontents

%%%%%%%%%%%%%%%%%%%%%%%%%%%%%%%%%%%%%%%%%%%%%%%%%%%%%%%%%%%%%%%%%%%%%%
%%%%%%%%%%%%%%%%%%%%%%%%%%%%%%%%%%%%%%%%%%%%%%%%%%%%%%%%%%%%%%%%%%%%%%
%%%%%%%%%%%%%%%%%%%%%%%%%%%%%%%%%%%%%%%%%%%%%%%%%%%%%%%%%%%%%%%%%%%%%%
\section{Theory}

As in the usual \ls, we imagine geometries to consist
of homogeneous regions $\{\mc R_r\}$. Homogeneous region 
$\{\mc R_r\}$ has relative dielectric permittivity 
$\epsilon_r.$ 

The boundary of $\mc R_r$ is denoted $\partial \mc R_r$; 
it may consist of a single closed surface or a union of
surfaces, each of which may be individually open.
Thus we write 
%====================================================================%
\numeq{RrSs} { \partial \mc R_r = \cup \, \mathcal{S}_s }
%====================================================================%
Each surface $\mc S_s$ bounds precisely two regions; 
thus $\mc S_s$ appears on the RHS of equation (\ref{RrSs})
for two different values of $r$. 

On each surface lives a surface charge density 
$\sigma(\vb x)$.

Surface $\mc S_s$ may optionally be PEC, in which case it
must be assigned a fixed potential $V_s$ before the problem
can be solved. Physically, a PEC surface corresponds to
an infinitesimally thin conducting layer at the interface
between two dielectric regions, and the surface charge
on such a surface represents free charges supplied to the
surface by whatever batteries initially charged them up to
their specified  potentials $V_s.$

On the other hand, a non-PEC surface simply desribes the 
interface between two dielectric regions, and the 
surface charge on such a surface represents the divergence
of the bound volume polarization density in the dielectric  
region. Among other things, this means that the total
surface charge integrated over the boundary of a dielectric
region must vanish. There is no such constraint on 
the total surface charge on PEC surfaces.

%====================================================================%
%====================================================================%
%====================================================================%
\subsection*{Potentials and fields from charge densities: 
             continuous forms} 

Let $\vb x$ be a point in a region $\mc R_r$. The electrostatic potential 
and field at $\vb x$ are obtained by summing contributions from
surface charges on all surfaces bounding $\mc R_r$: 
%====================================================================%
\begin{subequations}
\begin{align}
\phi(\vb x) 
&= \phi_r\sups{ext}(\vb x) 
  +\frac{1}{4\pi\epsilon_0} 
   \oint
   \frac{1}{|\vb x - \vb x^\prime|}
           \,\sigma(\vb x^\prime) \, d\vb x^\prime
\\
%--------------------------------------------------------------------%
\vb E(\vb x) 
&= \vb E_r\sups{ext}(\vb x)
  +\frac{1}{4\pi\epsilon_0}
   \oint
   \frac{(\vb x-\vb x^\prime)}
        {|\vb x - \vb x^\prime|^3}
        \,\sigma(\vb x^\prime)\,d\vb x^\prime
\end{align}
\label{ContinuousEquations1}
\end{subequations}
%====================================================================%
where the integrals are over all surfaces in the problem\footnote{Note
that the integrals in (\ref{ContinuousEquations1}) are over 
\textit{all} surfaces in the problem, not just the surfaces bounding 
the region in which the evaluation point lies. This is in contrast
to the full-wave case, in which the corresponding surface integrals
range only over the surfaces bounding the medium in question.}
and where $\phi_r\sups{ext},\vb E_r\sups{ext}$ are the potential
and field arising from external field sources lying inside
region $\mc R_r.$\footnote{In a BEM scattering problem these 
would be the ``incident'' fields, but in an electrostatic problem 
the word ``incident'' doesn't quite make sense, so we call them 
the ``external'' fields instead.} 

Note that equations (\ref{ContinuousEquations1}) do \textit{not}
involve the dielectric constants of the various regions. Instead,
surface charges on all surfaces contribute to the potential as
they would in vacuum.

%====================================================================%
%====================================================================%
%====================================================================%
\subsection*{Surface charge density expansion}

Now imagine approximating surface $\mc S_s$ as the union
of $N_s\supt{P}$ flat triangular panels, 
$$ \mc S_{s} = \cup \, \mc P_{sa}$$
where $a=1,\cdots,N_s\supt{P}$.
Let $P_{sa}$ have area $A_{sa}$ and surface normal
$\vbhat{n}_{sa}$. (We will worry about the direction
of $\vbhat{n}_{sa}$ later.)

To panel $\mc P_{sa}$ we assign a scalar-valued 
``pulse'' basis function $b_{sa}(\vb x)$ that is 
1 on the panel and 0 elsewhere:
$$ b_{sa}(\vb x) = 
   \begin{cases}
    1, \qquad &\vb x \in \mc P_{sa} \\
    0, \qquad &\text{otherwise}.
  \end{cases}
$$
We approximate the surface charge density on $\mc S_s$
as an expansion in the $b_{sa}$ functions:
$$ \frac{\sigma(\vb x)}{\epsilon_0} 
   \approx \sum_{a=1}^{N_s\supt{P}} \sigma_{sa} b_{sa}(\vb x)
   \qquad 
   \text{for }\vb x\in \mc S_s.
$$
Note that my $\sigma_{sa}$ unknowns have the dimensions
of 
$$\frac{\text{surface charge density}}{\text{permittivity}}=
  \frac{\text{volts}}{\text{length}}.
$$
In what follows, I will frequently use the collective
subscript $n=(sa)$ with $\sum_n=\sum_s \sum_a$, i.e.
a sum over $n$ runs over all panels on all surfaces.

%%%%%%%%%%%%%%%%%%%%%%%%%%%%%%%%%%%%%%%%%%%%%%%%%%%%%%%%%%%%%%%%%%%%%%
%%%%%%%%%%%%%%%%%%%%%%%%%%%%%%%%%%%%%%%%%%%%%%%%%%%%%%%%%%%%%%%%%%%%%%
%%%%%%%%%%%%%%%%%%%%%%%%%%%%%%%%%%%%%%%%%%%%%%%%%%%%%%%%%%%%%%%%%%%%%%
\subsection*{Potentials and fields from charge densities: 
             discretized forms} 

The electrostatic potential and field at $\vb x$ are 
%====================================================================%
\begin{subequations}
\begin{align}
\phi(\vb x) 
&= \phi_r\sups{ext}(\vb x)
   +
   \sum_{n} \sigma_n 
   \int_{\mc P_n} \frac{d\vb x^\prime}{4\pi |\vb x - \vb x^\prime|}
\\
%--------------------------------------------------------------------%
\vb E(\vb x) 
&= \vb E_r\sups{ext}(\vb x)
   +
   \sum_{n} \sigma_n 
   \int_{\mc P_n} \frac{(\vb x-\vb x^\prime)d\vb x^\prime}
                       {4\pi |\vb x - \vb x^\prime|^3}
\end{align}
\label{DiscreteEquations1}
\end{subequations}
where the sum is over all panels on all surfaces in the problem.
%====================================================================%

%%%%%%%%%%%%%%%%%%%%%%%%%%%%%%%%%%%%%%%%%%%%%%%%%%%%%%%%%%%%%%%%%%%%%%
%%%%%%%%%%%%%%%%%%%%%%%%%%%%%%%%%%%%%%%%%%%%%%%%%%%%%%%%%%%%%%%%%%%%%%
%%%%%%%%%%%%%%%%%%%%%%%%%%%%%%%%%%%%%%%%%%%%%%%%%%%%%%%%%%%%%%%%%%%%%%
\subsection*{Conditions on potentials and fields}

\subsubsection*{Conditions at PEC surfaces}

At PEC surfaces we impose the condition that the electrostatic
potential equal the specified potential for that conductor:
%====================================================================%
\begin{align}
 \phi(\vb x) &= V_s, \qquad \text{for } \vb x \in \mc S_s.
\nonumber
\intertext{Galerkin-testing with the expansion functions for 
           surface $\mc S_s$, we find }
 \int_{\mc P_{sa}} \phi(\vb x) d\vb x &= A_{sa} V_s,
 \qquad \text{for all panels $\mc P_{sa}$ on surface }\mc S_s.
\nonumber
 \intertext{Inserting (\ref{DiscreteEquations1}a), this reads }
 \sum_{n} \mathcal{I}\supt{(1)}_{mn} \sigma_n
 &= A_{m} V_m - \int_{\mc P_m} \phi\sups{ext}(\vb x)d\vb x
\label{PECPanelEquation}
\end{align}
where $V_m$ is the potential at which the conducting surface
containing panel $\mc P_m$ is held and
$$ \mathcal{I}\supt{(1)}_{mn} 
   \equiv 
   \int_{\mc P_m} \int_{\mc P_n} 
   \frac{1}{4\pi|\vb x - \vb x^\prime|} \, d\vb x^\prime \,d\vb x.
$$ 
%====================================================================%

\subsubsection*{Conditions at dielectric surfaces}

At non-PEC surfaces we impose the condition that the normal 
electric field exhibit the requisite discontinuity. 
If $\vb x$ is a point on a surface $\mc S_s$ lying between
regions $\mc R_r$ and $\mc R_{r^\prime}$, the condition is 
\numeq{EFieldDiscontinuity}
{
 \epsilon_r \left|\pard{\phi}{\vbhat{n}}\right|_{\vb x^+}
=
 \epsilon_{r^\prime} \left|\pard{\phi}{\vbhat{n}}\right|_{\vb x^-}
}
where $\vbhat{n}$ is the surface normal pointing 
away from $\mc R_r$ into $\mc R_{r^\prime}$, and
where $x^+$ and $x^-$ are points lying
infinitesimally displaced from $\vb x$ along
$\vbhat{n}$ into $\mc R_r$ and $\mc R_{r^\prime}$.

When we seek to enforce condition (\ref{EFieldDiscontinuity}) 
at a point $\vb x$ lying within a panel $\mc P_{sa}$
on $\mc S_s$, we find the following dichotomy:
%====================================================================%
\begin{enumerate}
  \item Surface charges on $\mc P_{sa}$ contribute to 
        the two sides of (\ref{EFieldDiscontinuity})
        with \textit{opposite} signs.
  \item Surface charges on all other panels, as well as the 
        external field soures, contribute
        to the two sides of (\ref{EFieldDiscontinuity})
        with the \textit{same} sign.
\end{enumerate}
Equation (\ref{EFieldDiscontinuity}) thus reads, for a 
point $\vb x$ on $\mc P_m$,
%====================================================================%
\begin{align*}
&\epsilon_r 
   \left[ \frac{\sigma_m}{2} 
          + \sum_{n\ne m} \sigma_n \int_{\mc P_{n}}
            \frac{ \vbhat{n}_m \cdot (\vb x-\vb x^\prime)}
                 { 4\pi |\vb x-\vb x^\prime|^3 } \, d\vb x^\prime
          + \vbhat{n}_m \cdot \vb E\sups{ext}(\vb x)
   \right]
\\
&\qquad
=\epsilon_{r^\prime}
   \left[  -\frac{\sigma_m}{2} 
          + \sum_{n\ne m} \sigma_n \int_{\mc P_{n}}
            \frac{ \vbhat{n}_m \cdot (\vb x-\vb x^\prime)}
                 { 4\pi |\vb x-\vb x^\prime|^3 } \, d\vb x^\prime
          + \vbhat{n}_m \cdot \vb E\sups{ext}(\vb x)
   \right]
\end{align*}
or 
\numeq{DielectricCondition}
{ \sigma_m 
  + \Delta_{rr^\prime}
    \sum_{n\ne m} \sigma_n \int_{\mc P_{n}}
    \frac{ \vbhat{n}_m \cdot (\vb x-\vb x^\prime)}
         { 4\pi |\vb x-\vb x^\prime|^3 } \, d\vb x^\prime
   = -\Delta_{rr^\prime}\vbhat{n}_m \cdot \vb E\sups{ext}(\vb x)
}
with
\numeq{DeltaDef}
{\Delta_{rr^\prime} = 2\frac{\epsilon_r-\epsilon_r^\prime}{\epsilon_r + \epsilon_r^\prime}.}
Now Galerkin-test (\ref{DielectricCondition}) with the pulse basis
function associated with panel $m$:
\numeq{DielectricPanelEquation}
{
   A_{m} \sigma_m 
   + \Delta_{rr^\prime} \sum_{n\ne m} \mathcal{I}\sups{(2)}_{mn} \sigma_n
   = -\Delta_{rr^\prime} \int_{\mc P_m} \vbhat{n}_m \cdot \vb E\sups{ext}(\vb x) d\vb x
}
where 
$$ \mathcal{I}\supt{(2)}_{mn} 
   \equiv 
   \int_{\mc P_m} \int_{\mc P_n} 
   \frac{\vbhat{n}_m \cdot (\vb x-\vb x^\prime)}
        {4\pi|\vb x - \vb x^\prime|^3} \, d\vb x^\prime \,d\vb x.
$$ 

\subsubsection*{BEM system}

Assembling equations (\ref{PECPanelEquation}) for all PEC panels
and (\ref{DielectricPanelEquation}) for all dielectric panels 
into a big linear system, we have 
%====================================================================%
\numeq{BEMSystem}{ \vb M \boldsymbol{\sigma} = \vb v }
%====================================================================%
where the $m$th entry of the unknown vector $\sigma$ is the 
surface charge density (divided by $\epsilon_0$) on the $m$th
panel, and where the elements of the BEM matrix and the RHS vector are
%====================================================================%
\paragraph{If panel $m$ is PEC:}
$$
  M_{mn} = \mathcal{I}\sups{(1)}_{mn}, 
   \hspace{1.2in}
   v_m = A_{m} V_m - \int_{\mc P_m} \phi\sups{ext}(\vb x) \, d\vb x
$$
\paragraph{If panel $m$ is dielectric:}
$$ M_{mn} = 
   \begin{cases} 
     A_m, \qquad &m=n \\
     \Delta_{rr^\prime} \mathcal{I}\sups{(2)}_{mn}, \qquad &m\ne n
   \end{cases} 
   \qquad 
   v_m = -\Delta_{rr^\prime} \int_{\mc P_m} \vbhat{n}_m \cdot \vb E\sups{ext}(\vb x) \, d\vb x.
$$
%====================================================================%
In these equations, 
\begin{itemize}
 \item $V_m$ denotes the potential at which the conducting surface
       containing panel $\mc P_m$ is held.
 \item $\Delta_{rr^\prime}$ is quantity (\ref{DeltaDef}) with $\epsilon_r$
       and $\epsilon_{r^\prime}$ the permittivities of the regions 
       \textit{exterior} and \textit{interior} to the surface containing
       $\mc P_m,$ respectively.
\end{itemize}

%%%%%%%%%%%%%%%%%%%%%%%%%%%%%%%%%%%%%%%%%%%%%%%%%%%%%%%%%%%%%%%%%%%%%%
%%%%%%%%%%%%%%%%%%%%%%%%%%%%%%%%%%%%%%%%%%%%%%%%%%%%%%%%%%%%%%%%%%%%%%
%%%%%%%%%%%%%%%%%%%%%%%%%%%%%%%%%%%%%%%%%%%%%%%%%%%%%%%%%%%%%%%%%%%%%%


\end{document}
