%%%%%%%%%%%%%%%%%%%%%%%%%%%%%%%%%%%%%%%%%%%%%%%%%%%%%%%%%%%%%%%%%%%%%%
%%%%%%%%%%%%%%%%%%%%%%%%%%%%%%%%%%%%%%%%%%%%%%%%%%%%%%%%%%%%%%%%%%%%%%
%%%%%%%%%%%%%%%%%%%%%%%%%%%%%%%%%%%%%%%%%%%%%%%%%%%%%%%%%%%%%%%%%%%%%%
\newpage
\section{Fields from surface currents}

To fix notation and ideas for the main text, in this section
I review the procedure for computing the fields from the 
surface currents in an SIE solver.

In the surface-integral-equation (SIE) approach to
classical electromagnetism problems, we solve for tangential
\textit{equivalent surface currents} (both electric currents $\vb K$ 
and magnetic currents $\vb N$) flowing on the surfaces of 
homogeneous material bodies. In numerical solvers, we 
approximate these as finite expansions in a discrete set
of $N\subt{B}$ basis functions; using a convenient 6-vector notation 
in which $\bmc C\equiv {\vb K \choose \vb N}$, we put
%--------------------------------------------------------------------%
$$ \bmc C(\vb x)=\sum_{\alpha=1}^{N\subt{B}} 
   c_\alpha \bmc B_\alpha(\vb x) 
$$ 
%--------------------------------------------------------------------%
where $\{\bmc B_\alpha\}$ is a set of 6-vector basis 
functions\footnote{In {\sc scuff-neq} these take the form
$\bmc B_\alpha=$ 
$\vb b_\alpha \choose 0$ 
or 
$0 \choose \vb b_\alpha$
where $\vb b_\alpha$ is a three-vector RWG basis function.
However, the FSC formalism is not specific to this choice.}
and $\{c_\alpha\}$ are scalar expansion coefficients.
(We work at a fixed frequency $\omega$ and assume all fields
and currents vary in time like $e^{-i\omega t}$.)

The electric and magnetic fields produced by these currents
are linear functions of the $\{c_\alpha\}$ coefficients.
In 6-vector notation with $\bmc F\equiv {\vb E \choose \vb H}$, we have
%--------------------------------------------------------------------%
\numeq{FFromC}
{ \bmc F(\vb x)=
   \sum_{\alpha} c_\alpha \bmc F_\alpha(\vb x)
}
%--------------------------------------------------------------------%
where $\bmc F_\alpha={\vb E_\alpha \choose \vb H_\alpha}$ 
is the six-vector of fields produced by basis function $\bmc B_\alpha$ 
populated with unit strength. These may be calculated numerically
as convolutions over basis functions,
%--------------------------------------------------------------------%
$$ \bmc F_\alpha(\vb x)
   =\int_{\sup \bmc B_\alpha} \bmc{G}(\vb x, \vb x^\prime)
    \bmc B_\alpha(\vb x^\prime) \, d\vb x^\prime
$$
%--------------------------------------------------------------------%
where $\bmc{G}$ is an appropriate $6\times 6$ dyadic Green's
function.
