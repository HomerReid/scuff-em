\documentclass[letterpaper]{article}

/home/homer/work/scuff-em/tex/scufftex.tex
%\usepackage{hyperref}

\graphicspath{{figures/}}

%------------------------------------------------------------
%------------------------------------------------------------
%- Special commands for this document -----------------------
%------------------------------------------------------------
%------------------------------------------------------------
\newcommand{\GB}{\overline{G}}
\newcommand{\vbGB}{\overline{\vb G}}

%------------------------------------------------------------
%------------------------------------------------------------
%- Document header  -----------------------------------------
%------------------------------------------------------------
%------------------------------------------------------------
\title {Implementation of Ewald Summation in {\sc scuff-em}}
\author {Homer Reid}
\date {June 12, 2012}


%------------------------------------------------------------
%------------------------------------------------------------
%- Start of actual document
%------------------------------------------------------------
%------------------------------------------------------------

\begin{document}
\pagestyle{myheadings}
\markright{Homer Reid: Ewald Summation in {\sc scuff-em}}
\maketitle

\tableofcontents

%%%%%%%%%%%%%%%%%%%%%%%%%%%%%%%%%%%%%%%%%%%%%%%%%%%%%%%%%%%%%%%%%%%%%%
%%%%%%%%%%%%%%%%%%%%%%%%%%%%%%%%%%%%%%%%%%%%%%%%%%%%%%%%%%%%%%%%%%%%%%
%%%%%%%%%%%%%%%%%%%%%%%%%%%%%%%%%%%%%%%%%%%%%%%%%%%%%%%%%%%%%%%%%%%%%%
\newpage

\section{The Periodic Dyadic Green's Function}

Consider a two-dimensional lattice consisting of a set of 
lattice vectors $\{\vb L=(L_x, L_y)\}.$
We use the symbol $\vb p=(p_x, p_y)$ to denote a two-dimensional
Bloch wavevector.

The Bloch-periodic version of the scalar Helmholtz Green's function is
%====================================================================%
\numeq{PGF}
{ \GB(k; \vb p; \vb x) 
  \equiv 
  \sum_{\vb L} e^{i\vb p \cdot \vb L} 
               \frac{ e^{ i k |\vb x + \vb L|}}
                    { 4\pi|\vb x + \vb L|}
}
%====================================================================%

\section{Evaluation by Ewald Summation: The JRS Method}

To evaluate the sum in (\ref{PGF}) efficiently, we use a method 
outlined by Jordan, Richter, and Sheng (JRS)\footnote{K. E. Jordan, G. R. Richter, 
and P. Sheng, "An Efficient Numerical Evaluation of the Green's 
Function for the Helmholtz Operator on Periodic Structures," 
\textit{Journal of Computational Physics} \textbf{63} 222 (1986).
\texttt{http://dx.doi.org/10.1016/0021-9991(86)90093-8}}
based on the original ideas of Ewald\footnote{}. For completeness,
I will briefly recapitulate the key ideas of this approach.

The starting point is the identity
$$ \frac{e^{ikr}}{4\pi r} 
   \equiv 
   \frac{1}{2\pi^{3/2}} \int_0^\infty e^{ -s^2 r^2 + \frac{k^2}{4s^2}} ds.
$$ 

\subsection*{Split the sum into direct-lattice-local and reciprocal-lattice-local
             contributions}

\begin{align}
\GB(k; \vb p; \vb x) 
 &= \GB^{(1)}(k; \vb p; \vb x) 
   +\GB^{(2)}(k; \vb p; \vb x) 
\label{GBDecomposition}
\\
\GB^{(1)}(k; \vb p; \vb x) 
&= \frac{1}{2\pi^{3/2}}
   \sum_{\vb L} e^{i\vb p \cdot \vb L} \int_0^E e^{-s^2 |\vb x+\vb L|^2 + \frac{k^2}{4s^2}} ds
\label{GB1Sum}
\\
\GB^{(2)}(k; \vb p; \vb x) 
&= \frac{1}{2\pi^{3/2}}
   \sum_{\vb L} e^{i\vb p \cdot \vb L} \int_E^\infty e^{-s^2 |\vb x + \vb L|^2 + \frac{k^2}{4s^2}} ds
\label{GB2Sum}
\end{align}

\subsection*{Evaluate $\GB^{(1)}$ in real space} 

The first step is to tackle the sum in (\ref{GB1Sum}) head-on. 
The integral in that equation may be evaluated in closed form in terms of
the complex error function:
\numeq{ErfInt}
{
   \int_0^E e^{-s^2 r^2 + \frac{k^2}{4s^2}} ds
   = 
}
Plugging into (\ref{GB1Sum}), the real-space sum is 
\begin{align*}
\GB^{(1)}(k; \vb p; \vb x) 
&= \frac{1}{2\pi^{3/2}}
   \sum_{\vb L}
\end{align*}
This sum converges very rapidly in real space. Indeed, as soon as 
$|\vb r + \vb L|$ is greater than a few times $E$, the error-function
factors start to bite down \textit{doubly exponentially} and
the contribution of distant lattice sites becomes utterly negligible.
In practice, for a 2D lattice we achieve 8 or more digits of accuracy 
by retaining only $\approx$ terms in the sum.
Thus we consider this term done, and move on to $\GB^{(2)}.$ 

\subsection*{Evaluate $\GB^{(2)}$ in reciprocal space} 

We would next like to do for $\GB^{(2)}$ what we just did for $\GB^{(1)}$,
but this requires a little more work. 

First, the integral in (\ref{GB2Sum}) involves the upper range of the 
original integration limit (from $E$ to $\infty$), but the nice closed-form 
expression we have, equation (\ref{ErfInt}), is for the lower range 
of the integration range (0 to $E$). We can get from one to the other 
by changing variables in (\ref{GB2Sum}) from $s$ to $1/s$: 

\begin{align*}
\int_E^\infty e^{-s^2 |\vb x + \vb L|^2 + \frac{k^2}{4s^2}} ds
&= \int_0^\frac{1}{E} \frac{1}{s^2} e^{-\frac{1}{s^2} |\vb x + \vb L|^2 + \frac{s^2 k^2}{4}} ds
\label{GB2Sum2}
\end{align*}

Now our integral looks the way we want it to

\end{document}
