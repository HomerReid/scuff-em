\documentclass{article} 
/home/homer/work/scuff-em/tex/scufftex.tex

\graphicspath{{figures/}}

%%%%%%%%%%%%%%%%%%%%%%%%%%%%%%%%%%%%%%%%%%%%%%%%%%%%%%%%%%%%%%%%%%%%%%
% special commands for this document %%%%%%%%%%%%%%%%%%%%%%%%%%%%%%%%% 
%%%%%%%%%%%%%%%%%%%%%%%%%%%%%%%%%%%%%%%%%%%%%%%%%%%%%%%%%%%%%%%%%%%%%%

%**************************************************
%* Document header info ***************************
%**************************************************
\title{Surface Impedance Boundary Conditions in {\sc scuff-em}}
\author {Homer Reid}
\date {May 9, 2012}

%**************************************************
%* Start of actual document ***********************
%**************************************************

\begin{document}
\maketitle

\pagestyle{myheadings}
\markright{Homer Reid: Surface Impedance Boundary Conditions in {\sc scuff-em} }

\tableofcontents 

%%%%%%%%%%%%%%%%%%%%%%%%%%%%%%%%%%%%%%%%%%%%%%%%%%%%%%%%%%%%%%%%%%%%%%
%%%%%%%%%%%%%%%%%%%%%%%%%%%%%%%%%%%%%%%%%%%%%%%%%%%%%%%%%%%%%%%%%%%%%%
%%%%%%%%%%%%%%%%%%%%%%%%%%%%%%%%%%%%%%%%%%%%%%%%%%%%%%%%%%%%%%%%%%%%%%
\newpage
\section{The Surface-Impedance Boundary Condition}

The usual boundary condition imposed at the surface of a
perfectly electrically conducting (PEC) scatterer is that
the total tangential electric field vanish:
%====================================================================%
\numeq{PECBC1}
 { \vb E_{\parallel}\sups{tot}(\vb x) = 0. }
%====================================================================%

At the surface of an \textit{imperfectly} electrically conducting
(IPEC) scatterer with dimensionless relative surface impedance
$\zeta$, the boundary condition (\ref{PECBC1}) is modified to
read 
%====================================================================%
\numeq{IPECBC1}
{
\vb E_{\parallel}\sups{tot}(\vb x)
= \zeta Z_0 \vbhat{n}\times \vb H\sups{tot}(\vb x)
}
%====================================================================%
where $Z_0\approx 377\,\Omega$ is the impedance of vacuum.

I will refer to (\ref{IPECBC1}) as the ``impedance boundary condition'' (IBC).

%%%%%%%%%%%%%%%%%%%%%%%%%%%%%%%%%%%%%%%%%%%%%%%%%%%%%%%%%%%%%%%%%%%%%%
%%%%%%%%%%%%%%%%%%%%%%%%%%%%%%%%%%%%%%%%%%%%%%%%%%%%%%%%%%%%%%%%%%%%%%
%%%%%%%%%%%%%%%%%%%%%%%%%%%%%%%%%%%%%%%%%%%%%%%%%%%%%%%%%%%%%%%%%%%%%%
\newpage
\section{Two SIE formulations for IPEC bodies}

\subsection{Review: SIE formulation for PEC bodies}

I will consider two distinct SIE formulations for IPEC bodies.
These are both variants of the usual SIE procedure for PEC
bodies, which---by way of review---I summarize thusly:
%
\begin{enumerate}
 \item We introduce an electric surface current $\vb K(\vb x)$ on 
       the surface of a PEC scatterer. This current is related to 
       the total tangential $\vb H$-field according to 
       \numeq{KDef}
       {\vb K(\vb x) = \vbhat{n}\times \vb H\sups{tot}(\vb x).}
 \item We do \textit{not} need to introduce a magnetic surface 
       current; such a current would be proportional to the total
       tangential $\vb E$ field, but this vanishes in view of 
       the boundary condition (\ref{PECBC1}):
       \numeq{NVanishes}
        {\vb N(\vb x) = -\vbhat{n}\times \vb E\sups{tot}(\vb x)\equiv0.}
 \item $\vb K$ gives rise to scattered $\vb E$ and $\vb H$ fields according to
       \begin{align}
        \vb E\sups{scat} &= \int \BG_{\parallel}\supt{EE}(\vb x, \vb x^\prime) 
                            \cdot \vb K(\vb x^\prime) d\vb x^\prime
        \\
        \vb H\sups{scat} &= \int \BG_{\parallel}\supt{ME}(\vb x, \vb x^\prime) 
                            \cdot \vb K(\vb x^\prime) d\vb x^\prime.
       \end{align}
 \item We solve for $\vb K$ by demanding that the scattered field
       to which it gives rise satisfy the boundary condition
       (\ref{PECBC1}):
       \begin{align}
                \vb E_{\parallel}\sups{scat}(\vb x)
            &= -\vb E_{\parallel}\sups{inc}(\vb x)
        \label{PECBC2}
        \intertext{or} 
             \underbrace{
                \int \BG_{\parallel}\supt{EE}(\vb x, \vb x^\prime) 
                     \cdot \vb K(\vb x^\prime) d\vb x^\prime 
                        }_{\BG_{\parallel}\supt{EE} \star \vb K}
            &= - \vb E\sups{inc}(\vb x).
        \label{PECBC3}
        \end{align}
\end{enumerate}

%=================================================
%=================================================
%=================================================
\subsection{First SIE formulation for IPEC bodies}

My first SIE formulation for IPEC bodies 
%
\begin{enumerate}
 \item As in the PEC case, to each IPEC I continue to assign an electric
       surface current $\vb K$ related to the total $\vb H$ field
       by equation (\ref{KDef}).
 \item As in the PEC case, I continue to assign \textit{no} magnetic 
       surface current $\vb N$ to IPEC surfaces.
 \item To determine $\vb K$ from $\vb E\sups{inc}$, I use 
        (\ref{IPECBC1}) in place of (\ref{PECBC1}) and (\ref{PECBC2}).
        This has the effect of replacing (\ref{PECBC3}) with
       %====================================================================%
       \numeq{IPECEquation1}
        {
                 \vbGamma\supt{EE} \star \vb K
      -\zeta Z_0\vbhat{n}\times \vbGamma\supt{ME} \star \vb K
                 = -\vb E\sups{inc} + \zeta Z_0 \vbhat{n}\times \vb H\sups{inc}
        }
       %====================================================================%
\end{enumerate}

%=================================================
%=================================================
%=================================================
\subsection{Second SIE formulation for IPEC bodies}

My alternative SIE formulation for IPEC bodies goes like this:
%
\begin{enumerate}
 \item As before, to each IPEC surface I continue to assign an electric
       surface current $\vb K$ related to the total $\vb H$ field 
       by equation (\ref{KDef}).
 \item Unlike before, I now assign to each IPEC
       surface a nonvanishing magnetic current, which is
       not an independent unknown but is instead
       determined by $\vb K$ via the (\ref{IPECBC1}):
       %====================================================================%
       $$\vb N(\vb x) = -\vbhat{n}\times \vb E\sups{tot}(\vb x)
                      = -\zeta Z_0 \vbhat{n}\times \vb K(\vb x).
       $$
       %====================================================================%
 \item $\vb K$ and $\vb N\equiv \vb N[\vb K]$ give rise to scattered $\vb E$ and 
       $\vb H$ fields according to
       \numeq{EHScat1}
       { \vb E\sups{scat} =
           \BG\supt{EE} \star \vb K
          +\BG\supt{EM} \star \vb N,
          \qquad
         \vb H\sups{scat} =
          \BG\supt{ME} \star \vb K
         +\BG\supt{MM} \star \vb N
       }
%       which I could write alternatively as 
%       \begin{align*}
%         \vb E\sups{scat} &= 
%           \int \left\{ \BG_{\parallel}\supt{EE}(\vb x, \vb x^\prime) 
%                        +Z_s \Big[\vbhat{n}\times \BG_{\parallel}\supt{ME}(\vb x, \vb x^\prime)\Big]
%                \right\}
%                        \cdot \vb K(\vb x^\prime) \,d\vb x^\prime
%         \\
%         \vb H\sups{scat} &= 
%           \int \left\{ \BG_{\parallel}\supt{ME}(\vb x, \vb x^\prime) 
%                        +Z_s \Big[\vbhat{n}\times \BG_{\parallel}\supt{MM}(\vb x, \vb x^\prime)\Big]
%                \right\}
%                        \cdot \vb K(\vb x^\prime) \,d\vb x^\prime
%       \end{align*}
 \item I solve for $\vb K$ by demanding that the scattered fields
       (\ref{EHScat1})
       satisfy the boundary condition (\ref{IPECBC1}):
       $$
           \vb E_{\parallel}\sups{scat}(\vb x) 
            -\zeta Z_0 \vbhat{n}\times \vb H_{\parallel}\sups{scat}(\vb x)
       = 
          -\vb E_{\parallel}\sups{inc}(\vb x) 
            +\zeta Z_0 \vbhat{n}\times \vb H_{\parallel}\sups{inc}(\vb x).
       $$
       or
       \begin{align*}
       \int\bigg\{ &\BG_{\parallel}\supt{EE}(\vb x, \vb x^\prime) 
                   \cdot \vb K(\vb x^\prime)
                   \\
                   &-\zeta Z_0\BG_{\parallel}\supt{ME}(\vb x, \vb x^\prime)
                   \cdot \Big[\vbhat{n}\times \vb K(\vb x^\prime)\Big] 
                   \\
                   &-\zeta Z_0\vbhat{n}\times \BG_{\parallel}\supt{EM}(\vb x, \vb x^\prime) 
                   \cdot \vb K(\vb x^\prime)
                   \\
                   &+\zeta^2 Z_0^2
                   \vbhat{n}\times\BG_{\parallel}\supt{MM}(\vb x, \vb x^\prime)
                   \cdot \Big[\vbhat{n}\times \vb K(\vb x^\prime)\Big]
       \bigg\}
       = -\vb E_{\parallel}\sups{inc}(\vb x) 
            +\zeta Z_0 \vbhat{n}\times \vb H_{\parallel}\sups{inc}(\vb x).
       \end{align*}
\end{enumerate}

\end{document}
