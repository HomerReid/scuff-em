%%%%%%%%%%%%%%%%%%%%%%%%%%%%%%%%%%%%%%%%%%%%%%%%%%%%%%%%%%%%%%%%%%%%%%
%%%%%%%%%%%%%%%%%%%%%%%%%%%%%%%%%%%%%%%%%%%%%%%%%%%%%%%%%%%%%%%%%%%%%%
%%%%%%%%%%%%%%%%%%%%%%%%%%%%%%%%%%%%%%%%%%%%%%%%%%%%%%%%%%%%%%%%%%%%%%
\newpage
\section{Energy/momentum-transfer PFT (EMTPFT), version 1}

In the EMTPFT approach, we compute the power, force and torque
on a body $\mc B$ by considering the transfer of energy and
momentum from the total fields to the equivalent surface currents:
%====================================================================%
\begin{subequations}
\begin{align}
 P\sups{abs}&=\frac{1}{2}\text{Re }\int \bmc C^* \cdot \bmc F \, dV 
\\
 F_i&=\frac{1}{2\omega}\text{Im }\int \bmc C^* \cdot \partial_i \bmc F \, dV
\\
 \mc T_i
 &=\frac{1}{2\omega}\text{Im }\int
 \underbrace{
 \Big[   \bmc C^* \times \bmc F 
       + \bmc C^* \cdot \partial_{\theta_i} \vb F\Big]
            }_{\bmc C^* \cdot \wt \partial_i \bmc F}dV
\end{align}
\label{EMTPFT}%
\end{subequations}%
%====================================================================%
Equation (\ref{EMTPFT}a) is just the usual Joule heating
$P=\frac{1}{2}\text{Re } \big(\vb J^*\cdot \vb E + \vb M^* \cdot \vb H\big)$,
while Equations (\ref{EMTPFT}b,c) follow from considering the time-average
Lorentz force 
$d\vb F=\frac{1}{2}\text{Re }
      (\rho\subt{E}^* \vb E + \mu\, \vb J^* \! \times \! \vb H
      +\rho\subt{M}^* \vb H - \epsilon\, \vb M^* \! \times \! \vb E
      )dV
$
and torque $\vb r\times d\vb F$ on the charges and currents
in an infinitesimal volume $dV$; integrating over the
volume and using integration by parts and Maxwell's equations
yields (\ref{EMTPFT}b,c). [The symbol $\partial_{\theta_i}$
in (\ref{EMTPFT}c) denotes differentiation with 
respect to infinitesimal rotation about the $i$th coordinate 
axis. The symbol $\wt{\partial_i}$ is shorthand for the 
operation (cross product plus angular derivative)
involved in (\ref{EMTPFT}c).]

In what follows it will be convenient to express equations
(\ref{EMTPFT}) in terms of the following shorthand operator notation:
%====================================================================%
\numeq{EMTPFTShorthand}
{
 Q=\frac{1}{2}\int_{\partial \mc B}\bmc C^* \circledast \bmc F \, dA
}
%====================================================================%
where $Q=\{P\sups{abs}, F_i, \mc T_i\}$ and
the operator $\circledast$ correspondingly operates on $\bmc F$ as
in equations (\ref{EMTPFT}a,b,c).

Equations (\ref{EMTPFT}) involve the total fields at the
body surface. Because SIE formulations allow surface fields
to be computed in either of two distinct ways---namely, as the
limiting values of the bulk fields as one approaches the
surface from the exterior or the interior of the body---the EMTPFT
may actually be computed in two difference ways:
%====================================================================%
\begin{subequations}
\begin{align}
 Q &=   \int \bmc C^* \circledast \bmc F\sups{inc,ext} d\vb x
      + \iint \bmc C^* \circledast \bmc G\sups{ext} \bmc C\,d\vb x \,d\vb x^\prime
\\
   &=   \int \bmc C^* \circledast \bmc F\sups{inc,int} d\vb x
      - \iint \bmc C^* \circledast \bmc G\sups{int} \bmc C\,d\vb x \,d\vb x^\prime
\end{align}
\label{EMTPFTChoice}
\end{subequations}
%====================================================================%
Here $\bmc F\sups{inc,ext}$ and $\bmc F\sups{inc,int}$ are
the contributions of incident-field sources lying external and internal
to the body and $\bmc G\sups{ext,int}$ are the Green's functions
for the exterior and interior media.
In both cases of (\ref{EMTPFTChoice}) I refer to the two
terms as the ``extinction'' and minus the ``scattered''
PFT:
%====================================================================%
\begin{align*}
 Q&=Q\sups{ext} - Q\sups{scat}
\\
 Q\sups{ext}
&=\int \bmc C^* \circledast \bmc F\sups{inc} \, d\vb x
\\
 Q\sups{scat}
&=\mp \iint \bmc C^* \circledast \bmc G \bmc C\,d\vb x \, d\vb x^\prime
\end{align*}
%====================================================================%
For the typical case of a body irradiated by external sources,
there is no extinction contribution to the interior EMTPFT, and 
thus in this case the ``scattered'' PFT is actually the full PFT.

\subsection{Extinction EMTPFT}

The contributions of the incident fields to the EMTPFT are
%====================================================================%
$$ Q\sups{ext}=
   \frac{1}{2}\sum_{\alpha} c_\alpha
   \int_{\sup \bmc B_\alpha} \bmc B_\alpha \circledast \bmc F\sups{inc} \, dA
$$
%====================================================================%
The integrals here are non-singular two-dimensional integrals over
the basis-function supports, with the integrand involving values 
and derivatives of the incident fields. These are evaluated 
in {\sc scuff-em} by simple low-order numerical cubature.

\subsection{Scattered EMTPFT}

The scattered fields at a point $\vb x_a$ are
%====================================================================%
\begin{align*}
 \vb E\sups{scat}(\vb x_a)
 &=i\omega \mu \sum_b k_b \vb e_b(\vb x_a) - \sum_b n_b \vb h_b(\vb x_a)
\\
 \vb H\sups{scat}(\vb x_a)
 &=\sum_b k_b \vb h_b(\vb x_a) + i\omega \epsilon \sum_b n_b \vb e_b(\vb x_a)
\end{align*}
where the sums are over all basis functions on all surfaces bounding
the region containing $\vb x_a$, and where the ``reduced fields'' of
an individual basis function are
%====================================================================%
\begin{align*}
 \vb e_b(\vb x_a)
 &= \int_{\sup \vb b_b}
    \Big[\vb b_b \phi + \frac{1}{k^2}(\nabla \cdot \vb b_b)\vb r \psi \Big]d\vb x_b
\\
 \vb h_b(\vb x_a)
 &= \int_{\sup \vb b_b}
    \Big( \vb r \times \vb b_b \Big) \psi \, d\vb x_b
\end{align*}
%====================================================================%
with 
$$ \vb r\equiv \vb x_a-\vb x_b, \qquad
  \phi\equiv \frac{e^{ikr}}{4\pi r}, \qquad
  \psi\equiv \frac{e^{ikr}}{4\pi r^3}(ikr-1).
$$
%====================================================================%
(Note that $\psi$ is defined such that $\nabla \phi = \vb r\psi.$)
The scattered contribution to PFT quantities then reads
%====================================================================%
\begin{align*}
Q\sups{scat}
&= \mp 
   \int\Big\{
    \vb K^*(\vb x_a) \circledast \vb E\sups{scat}(\vb x_a)
  + \vb N^*(\vb x_a) \circledast \vb H\sups{scat}(\vb x_a)
      \Big\} \, d\vb x
\\
&=\sum_{ab} \Big\{ \Big[\mu k_a^* k_b + \epsilon n_a^* n_b\Big]
                    Q\supt{KKPNN}_{ab}
                  +\Big[k_a^* n_b - n_a^* k_b\Big]
                    Q\supt{KNMNP}_{ab}
            \Big\}
\end{align*}
with
%====================================================================%
$$
Q\supt{KKPNN}_{ab} 
 \equiv i\omega \iint \vb b_a \circledast \vb e_b \, d\vb x_a \, d\vb x_b,
\qquad
Q\supt{KNMNK}_{ab} 
 \equiv -\iint \vb b_a \circledast \vb h_b \, d\vb x_a \, d\vb x_b
$$
%====================================================================%
\subsection*{Power}
%====================================================================%
\begin{align*}
 P\supt{KKPNN}_{ab} 
 &= i\omega \iint \Big[ \vb b_a \cdot \vb b_b \phi
                + \frac{(\nabla \cdot \vb b_b)}{k^2}
                  \vb b_a \cdot \nabla\phi \Big]
\,d\vb x_a \, d\vb x_b
\\
 &= i\omega 
    \iint \underbrace{\bigg[ \vb b_a \cdot \vb b_b 
               - \frac{(\nabla \cdot \vb b_a)(\nabla \cdot \vb b_b)}{k^2}
                      \bigg]
                     }_{P\supt{EFIE}} \phi
\,d\vb x_a \, d\vb x_b
\\
 P\supt{KNMNK}_{ab} 
 &=-\iint \Big[ \vb b_a \cdot (\vb r \times \vb b_b)\Big]\psi 
                \,d\vb x_a \, d\vb x_b
\\
 &=+\iint \underbrace{
          \Big[ \vb r \cdot (\vb b_a \times \vb b_b)\Big]
                     }_{P\supt{MFIE}}\psi
                \,d\vb x_a \, d\vb x_b
\\
\end{align*}
%====================================================================%

\subsection*{Force}

First note 
%====================================================================%
\begin{align*}
  \partial_i \vb e_b &=
  \int \Big\{ \vb b_b r_i \psi +
              \frac{\nabla\cdot \vb b_b}{k^2}\partial_i\Big(\vb r\psi\Big)
       \Big\}\, d\vb x_b
\\
  (\partial_i \vb h_b)_j
&=\partial_i \int \varepsilon_{jk\ell} \, r_k b_{b\ell} \, \psi \, d\vb x_b
\\
&=\partial_i \int \varepsilon_{jk\ell} \, r_k b_{b\ell} \, \psi \, d\vb x_b
\end{align*}
%====================================================================%

%====================================================================%
\begin{align*}
 (F_i\supt{KKPNN})_{ab} 
 &= i\omega \iint \vb b_a \cdot \partial_i \vb e_b \,d\vb x_a \, d\vb x_b
\\
 (F_i\supt{KNMNK})_{ab} 
 &= -\iint \vb b_a \cdot \partial_i \vb h_b \, d\vb x_a \, d\vb x_b
\end{align*}
%====================================================================%

In the exterior region, the total fields receive
incident and scattered contributions,
$\bmc F = \bmc F\sups{inc} + \bmc F\sups{scat}$,
We identify the corresponding contributions to the
PFT quantities (\ref{EMTPFT}) as respectively the
extinction PFT and the (negative of the) scattered PFT,
%====================================================================%
\begin{subequations}
\begin{align}
 Q &= Q\sups{ext} - Q\sups{scat}
\\
 Q\sups{ext}&=\frac{1}{2}\int_{\partial \mc B}\bmc C^* \circledast \bmc F\sups{inc},
\\
 Q\sups{scat}&=-\frac{1}{2}\int_{\partial \mc B}\bmc C^* \circledast \bmc F\sups{scat}
\end{align}
\label{ExtScat}%
\end{subequations}
%====================================================================%

\subsubsection*{Extinction contributions to exterior EMTPFT}

\subsubsection*{Scattered contributions to exterior EMTPFT}

The scattered-field contributions to the EMTPFT power and force 
follow from the discretized form of (\ref{ExtScat}c) and read
%====================================================================%
\begin{subequations}
\begin{align}
 P\sups{scat}&=-\frac{1}{2}\text{Re} \sum_{ab}
 \left(\begin{array}{c} k_a \\ n_a\end{array}\right)^\dagger
 \left(\begin{array}{cc}i\omega \mu \vb G_{ab} & \wh{\vb C}_{ab} \\
                        -\wh{\vb C}_{ab} & i\omega\epsilon \vb G_{ab}
 \end{array}\right)
 \left(\begin{array}{c}k_b \\ n_b\end{array}\right)
\\
 F_i\sups{scat}&=-\frac{1}{2\omega}\text{Im} \sum_{ab}
 \left(\begin{array}{c} k_a \\ n_a\end{array}\right)^\dagger
 \left(\begin{array}{cc}i\omega \mu \partial_i \vb G_{ab} & \partial_i \wh{\vb C}_{ab} \\
                        -\partial_i \wh{\vb C}_{ab} & i\omega\epsilon \partial_i \vb G_{ab}
 \end{array}\right)
 \left(\begin{array}{c}k_b \\ n_b\end{array}\right)
\end{align}
\label{PFExteriorPFT1}%
\end{subequations}
%====================================================================%
(The torque is similar to the force with $\partial_i \to \wt{\partial_i}$.)
In these equations, 
\begin{itemize}
\item
$a$ runs over all interior edges (RWG functions) 
on the surface bounding the region on which we are 
computing the power or force;
\item
$b$ runs over all interior edges on all surfaces bounding 
the region exterior to the region on which we are 
computing the power or force;
\item
$\epsilon=\epsilon_0 \epsilon^r$ and 
$\mu=\mu_0 \mu^r$ are the material properties of the 
exterior region
\item
 $\vb G_{ab}=\vmv{\vb b_a}{\mb G}{\vb b_b}$
 is the matrix element of the $\mb G$ kernel for
 the exterior medium between RWG basis functions.
\item
 $\wh{\vb C}_{ab}=\vmv{\vb b_a}{ik\mb C}{\vb b_b}.$
\end{itemize}

Upon using the relations
%====================================================================%
$$\{\vb G_{ba}, \vb C_{ba}\} = \{\vb G_{ab}, \vb C_{ab}\}, 
  \qquad
  \partial_i\{\vb G_{ba}, \vb C_{ba}\} = -\partial_i\{\vb G_{ab}, \vb C_{ab}\}, 
$$
%====================================================================%
equations (\ref{PFExteriorPFT1}) may be simplified to read
%====================================================================%
\begin{align*}
 P\sups{scat} &= +\frac{1}{2}
  \sum_{ab}\bigg\{
           \omega \mu_0
           \Big[\text{Re }k_a^* k_b\Big]
           \Big[\text{Im }\mu^r \vb G_{ab}\Big]
          +\omega \epsilon_0
           \Big[\text{Re }n_a^* n_b\Big]
           \Big[\text{Im }\epsilon^r \vb G_{ab}\Big]
\\
&\hspace{1in} 
  +\Big[\text{Im }k_a^* n_b - n_a^* k_b\Big]
           \Big[\text{Im } \wh{\vb C}_{ab}\Big]\bigg\}
\\
 F_i\sups{scat} &= +\frac{1}{2\omega}
  \sum_{ab}\bigg\{
           \omega \mu_0
           \Big[\text{Im }k_a^* k_b\Big]
           \Big[\text{Im }\mu^r \partial_i \vb G_{ab}\Big]
          +\omega \epsilon_0
           \Big[\text{Im }n_a^* n_b\Big]
           \Big[\text{Im }\epsilon^r \partial_i \vb G_{ab}\Big]
\\
&\hspace{1in} 
  -\Big[\text{Re }k_a^* n_b - n_a^* k_b\Big]
           \Big[\text{Im } \partial_i \wh{\vb C}_{ab}\Big]\bigg\}
\end{align*}
%====================================================================%

\subsubsection*{EMTPFT matrix elements}

%====================================================================%
\begin{align*}
   \partial_i \vb G_{ab}
&= \partial_i \vb G^1_{ab} +\partial_i \vb G^2_{ab}
\\
\partial_i \vb G^1_{ab}
&=
 \int_{\sup \vb b_a}\, d\vb x_a \, 
 \int_{\sup \vb b_b}\, d\vb x_b \, 
 \left(\vb b_a \cdot \vb b_b\right) \partial_i G_0(\vb x_a-\vb x_b)
\\
\partial_i \vb G^2_{ab}
&=
 -\frac{1}{k^2}
  \int_{\sup \vb b_a}\, d\vb x_a \,
  \int_{\sup \vb b_b}\, d\vb x_b \,
   (\nabla \cdot \vb b_a)(\nabla \cdot \vb b_b)
   \partial_i G_0(\vb x_a-\vb x_b)
\end{align*}
%====================================================================%

%====================================================================%
\begin{align*}
 \text{Im } \partial_i \wh{\mb C}_{jk}
&= \text{Im }
   \partial_i\left[ \frac{e^{ikr}}{4\pi r^3}(ikr-1)\varepsilon_{jk\ell}r_\ell\right]
\\
\end{align*}
%====================================================================%

%====================================================================%
\begin{align*}
\partial_i G_0(\vb r)
   &= r_i(ikr-1)\frac{e^{ikr}}{4\pi r^3}
   &= -\frac{r_i}{4\pi r^3} + \Big[\partial_i G_0\Big]\supt{DS}
\end{align*}
%====================================================================%

%====================================================================%
\begin{align*}
 F_i\supt{S}
 &\frac{\mu_0}{2} \text{Im }{k_a^* k_b}\Big[\text{Im }
\end{align*}
%====================================================================%

\subsection{Torque}

\begin{align*}
 \mc T_i 
&= \frac{1}{2\omega}\text{Im }
\int \Big\{\vb K^* \times \vb E + \vb K \cdot \partial_{\theta}\vb E
           +\Big( \{\vb K, \vb E\} \leftrightarrow \{\vb N, \vb H\}\Big)\Big\}
\end{align*}

\begin{align*}
\Big(\vb K^* \times \vb E + \vb K \cdot \partial_{\theta}\vb E\Big)_i 
&=
 \varepsilon_{ijk}\Big[ K^*_j E_k + K^*_\ell X_j \partial_k E_\ell\Big]
\qquad \Big(\vb X \equiv \vb x_a - \vb x_0\Big)
\end{align*}

%====================================================================%
%\begin{align*}
% \vb E(\vb x_a)
%&=
% \sum_b \Big\{ i\omega \mu k_b \vb e_b(\vb x_a) - n_b \vb h_b(\vb x_a)\Big\}
%\\
%&=
% \sum_b \Big\{ i\omega \mu k_b \Big[ \vb b_b \Phi
%\end{align*}
%====================================================================%

%&
%\sum_{ab}k^*a
% \varepsilon_{ijk}\Big[ b_{aj}\Big(
% + K^*_\ell X_j \partial_k E_\ell\Big]
%\end{align*}

\subsection{Interior EMTPFT}

%====================================================================%
\begin{align*}
F_i&=
 \frac{1}{2\omega}\text{Im } 
 \int_{\mc S}\bigg\{ \vb K^* \cdot \partial_i \vb E
                    +\vb N^* \cdot \partial_i \vb H
             \bigg\}\, dA
\end{align*}
%====================================================================%

\paragraph{Planar case}

Working in a coordinate system in which the
%====================================================================%
\begin{align*}
\vb F&=
 \left( K^*_x \partial_x E_x + K^*_y \partial_x E_y \right)\vbhat{x}
 +
 \left( K^*_x \partial_y E_x + K^*_y \partial_y E_y \right)\vbhat{y}
 +
 \left( K^*_x \partial_z E_x + K^*_y \partial_z E_y \right)\vbhat{z}
\end{align*}
%====================================================================%
I organize the various terms here as follows:
%====================================================================%
\begin{align}
\vb F
= &\hphantom{+}\underbrace{
   \left( K^*_x \partial_x E_x\right) \vbhat{x}
  +\left( K^*_y \partial_y E_y \right)\vbhat{y}
              }_{T_1}
\\
&+ \underbrace{
   \left( K^*_y \partial_x E_y\right) \vbhat{x}
  +\left( K^*_x \partial_y E_x \right)\vbhat{y}
              }_{T_2}
\\
&+ \underbrace{
    \left( K^*_x \partial_z E_x + K^*_y \partial_z E_y \right)\vbhat{z}
              }_{T_3}
\end{align}
%====================================================================%
I now add to $T_1$ and subtract from $T_2$ the quantity
%====================================================================%
$ \left( K^*_y \partial_y E_x\right) \vbhat{x}
  +\left( K^*_x \partial_x E_y \right)\vbhat{y}
$. The term $T_1$ becomes
%====================================================================%
\begin{align}
T_1 &=  \left( K^*_x \partial_x E_x + K^*_y \partial_y E_x\right) \vbhat{x}
       +\left( K^*_x \partial_x E_y + K^*_y \partial_y E_y \right)\vbhat{y}
\\
&\sim -\Big(\nabla_{\parallel} \cdot \vb K^*\Big) \vb E_{\parallel}
\intertext{(where $\parallel$ indicates surface-tangential vector components
           and $\sim$ indicates equality up to total-derivative terms
           that make no contribution to the surface integral). The term
           $T_2$ becomes}
T_2 &= 
   \left( K^*_y \partial_x E_y\right) \vbhat{x}
  +\left( K^*_x \partial_y E_x \right)\vbhat{y}
  -\left( K^*_y \partial_y E_x\right) \vbhat{x}
  -\left( K^*_x \partial_x E_y \right)\vbhat{y}
\\
&= \underbrace{\Big(K_y^* \vbhat{x} - K_x^* \vbhat{y}\Big)}
             _{\vb K^* \times \vbhat{z}}
   \cdot
   \underbrace{
   \Big(\partial_x E_y - \partial_y E_x\Big)
              }_{(\nabla \times \vb E)_z=i\omega \mu H_z} 
\end{align}
%====================================================================%
In term $T_3$ above I write $(\nabla \times \vb E = i\omega \mu \vb H)$
%====================================================================%
$$ \partial_z E_x = \partial_x E_z + i\omega\mu H_y, \qquad
   \partial_z E_y = \partial_y E_z - i\omega\mu H_x
$$
%====================================================================%
and obtain
%====================================================================%
\begin{align*}
 T_3 &= 
  \Big(K_x^* \partial_x E_z + K_y^* \partial_y E_z \Big)\vbhat{z}
 +i\omega \mu \Big( K_x^* H_y - K_y^* H_x\Big)\vbhat{z}
\end{align*}
%====================================================================%
Integrating by parts in the first term, I find
%====================================================================%
$$ T_3 \sim
   -\Big(\nabla \cdot \vb K^*\Big) E_z \vbhat{z}
   +i\omega \mu \times \Big(\vb K^* \times \vb H\Big)_z \vbhat{z}
$$
%====================================================================%
Adding it all up and including the contribution of the 
$\vb{N}^* \partial_i \vb H$ term, I have 
%====================================================================%
\begin{align*}
 F&=\frac{1}{2\omega}\text{Im }\int_{\mc S}
   \Big[ -\big(\nabla \cdot \vb K^*\big) \vb E
         -\big(\nabla \cdot \vb N^*\big) \vb H
        +i\omega \mu \Big(\vbhat{K}^* \times \vbhat{H}\Big)
        -i\omega \epsilon \Big(\vbhat{N}^* \times \vbhat{E}\Big)
   \Big]dA
\end{align*}
%====================================================================%
 
\paragraph{General case}
\begin{align*}
 K^*_j \partial_i E_j
&= \underbrace{K^*_j \hat{n}_i \hat{n}_k \partial_k E_j}_{T_1}
  +\underbrace{K^*_j \Big[\delta_{ik}-\hat{n}_i \hat{n}_k\Big] \partial_k E_j}_{T_2}
\end{align*}
%====================================================================%

In $T_1$ write the Maxwell equation
$\nabla \times \vb E = i\omega \mu \vb H$
in the form
$$ \partial_k E_j = \partial_j E_k + i\omega \mu \varepsilon_{\ell kj} H_\ell $$
to obtain
\begin{align*}
 T_1 &= \underbrace{K^*_j \hat{n}_i \hat{n}_k \partial_j E_k}_{T_{1a}}
       +\underbrace{i\omega \mu K^*_j \hat{n}_i \hat{n}_k \varepsilon_{\ell kj} H_\ell}_{T_{1b}}
\end{align*}

In $T_{1a}$ integrate by parts:
%====================================================================%
$$ T_{1a}\sim 
   \underbrace{-\big(\partial_j K^*_j\big)\hat n_i \hat n_k E_k}_{T_{1a1}}
  \quad
   -\underbrace{K^*_j E_k \partial_j \Big( \hat{n}_i \hat{n}_k \Big)}_{T_{1a2}}
$$
%====================================================================%
where $\sim$ indicates equality up to total derivative terms that yield
vanishing contributions to surface integrals.
(The first term here vanishes because it involves the normal derivative
of the tangential-vector quantity $\vb K^*$.)

On the other hand, $T_{1b}$ is just the quantity
%====================================================================%
$$ T_{1b} = i\omega \mu
   \Big[ \vbhat{n} \cdot \big(\vb K^* \times \vb H\big) \Big] \hat{n}_i 
$$
%====================================================================%


In $T_{2}$ integrate by parts:
%====================================================================%
$$ T_{2}\sim 
   \underbrace{-\Big(\partial_k K^*_j\Big) E_j \Big(\delta_{ik}-\hat{n}_i \hat{n}_k\Big)}_{T_{2a}}
 \quad
   \underbrace{-\Big(K_j^* E_j\Big)
 \partial_k \Big(\delta_{ik}-\hat{n}_i \hat{n}_k\Big)}_{T_{2b}}
$$
%====================================================================%
