%%%%%%%%%%%%%%%%%%%%%%%%%%%%%%%%%%%%%%%%%%%%%%%%%%%%%%%%%%%%%%%%%%%%%%
%%%%%%%%%%%%%%%%%%%%%%%%%%%%%%%%%%%%%%%%%%%%%%%%%%%%%%%%%%%%%%%%%%%%%%
%%%%%%%%%%%%%%%%%%%%%%%%%%%%%%%%%%%%%%%%%%%%%%%%%%%%%%%%%%%%%%%%%%%%%%
\newpage
\section{Energy/momentum-transfer PFT (EMTPFT), version 1}

In the EMTPFT approach, we compute the power, force and torque
on a body $\mc B$ by considering the transfer of energy and
momentum from the total fields to the equivalent surface currents:
%====================================================================%
\begin{subequations}
\begin{align}
 P\sups{abs}&=\frac{1}{2}\text{Re }\int \bmc C^* \cdot \bmc F \, dV 
\\
 F_i&=\frac{1}{2\omega}\text{Im }\int \bmc C^* \cdot \partial_i \bmc F \, dV
\\
 \mc T_i
 &=\frac{1}{2\omega}\text{Im }\int
 \underbrace{
 \Big[   \bmc C^* \times \bmc F 
       + \bmc C^* \cdot \partial_{\theta_i} \vb F\Big]
            }_{\bmc C^* \cdot \wt \partial_i \bmc F}dV
\end{align}
\label{EMTPFT}%
\end{subequations}%
%====================================================================%
Equation (\ref{EMTPFT}a) is just the usual Joule heating
$P=\frac{1}{2}\text{Re } \big(\vb J^*\cdot \vb E + \vb M^* \cdot \vb H\big)$,
while Equations (\ref{EMTPFT}b,c) follow from considering the time-average
Lorentz force 
$d\vb F=\frac{1}{2}\text{Re }
      (\rho\subt{E}^* \vb E + \mu\, \vb J^* \! \times \! \vb H
      +\rho\subt{M}^* \vb H - \epsilon\, \vb M^* \! \times \! \vb E
      )dV
$
and torque $\vb r\times d\vb F$ on the charges and currents
in an infinitesimal volume $dV$; integrating over the
volume and using integration by parts and Maxwell's equations
yields (\ref{EMTPFT}b,c). [The symbol $\partial_{\theta_i}$
in (\ref{EMTPFT}c) denotes differentiation with 
respect to infinitesimal rotation about the $i$th coordinate 
axis. The symbol $\wt{\partial_i}$ is shorthand for the 
operation (cross product plus angular derivative)
involved in (\ref{EMTPFT}c).]

In what follows it will be convenient to express equations
(\ref{EMTPFT}) in terms of the following shorthand operator notation:
%====================================================================%
\numeq{EMTPFTShorthand}
{
 Q=\frac{1}{2}\int_{\partial \mc B}\bmc C^* \circledast \bmc F \, dA
}
%====================================================================%
where $Q=\{P\sups{abs}, F_i, \mc T_i\}$ and
the operator $\circledast$ correspondingly operates on $\bmc F$ as
in equations (\ref{EMTPFT}a,b,c). Note that the $\text{Re, Im}$
operations in (\ref{EMTPFT}) are understood as included in 
the symbol $\circledast$.

Equations (\ref{EMTPFT}) involve the total fields at the
body surface. Because SIE formulations allow surface fields
to be computed in either of two distinct ways---namely, as the
limiting values of the bulk fields as one approaches the
surface from the exterior or the interior of the body---the EMTPFT
may actually be computed in two different ways:
%====================================================================%
\begin{subequations}
\begin{align}
 Q &= \frac{1}{2}  \int \bmc C^* \circledast \bmc F\sups{inc,ext} d\vb x
     +\frac{1}{2} \iint \bmc C^* \circledast \bmc G\sups{ext} \bmc C\,d\vb x \,d\vb x^\prime
\\
   &= \frac{1}{2}  \int \bmc C^* \circledast \bmc F\sups{inc,int} d\vb x
     -\frac{1}{2} \iint \bmc C^* \circledast \bmc G\sups{int} \bmc C\,d\vb x \,d\vb x^\prime
\end{align}
\label{EMTPFTChoice}
\end{subequations}
%====================================================================%
Here $\bmc F\sups{inc,ext}$ and $\bmc F\sups{inc,int}$ are
the contributions of incident-field sources lying external and internal
to the body and $\bmc G\sups{ext,int}$ are the Green's functions
for the exterior and interior media.
In both cases of (\ref{EMTPFTChoice}) I refer to the two
terms as the ``extinction'' and minus the ``scattered''
PFT:
%====================================================================%
\begin{align*}
 Q&=Q\sups{ext} - Q\sups{scat}
\\
 Q\sups{ext}
&=\frac{1}{2}\int \bmc C^* \circledast \bmc F\sups{inc} \, d\vb x
\\
 Q\sups{scat}
&=\mp \frac{1}{2} \iint \bmc C^* \circledast \bmc G \bmc C\,d\vb x \, d\vb x^\prime
\end{align*}
%====================================================================%
For the typical case of a body irradiated by external sources,
there is no extinction contribution to the interior EMTPFT, and 
thus in this case the ``scattered'' PFT is actually the full PFT.

\subsection{Extinction EMTPFT}

The contributions of the incident fields to the EMTPFT are
%====================================================================%
$$ Q\sups{ext}=
   \frac{1}{2}\sum_{\alpha} c_\alpha
   \int_{\sup \bmc B_\alpha} \bmc B_\alpha \circledast \bmc F\sups{inc} \, dA
$$
%====================================================================%
or, more specifically,
%====================================================================%
\begin{align*}
 P\sups{ext} &=
 \frac{1}{2}\text{Re }\sum_{a}
  \Big\{  k_a^* \Expval{\vb b_a \cdot \vb E\sups{inc}}
         +n_a^* \Expval{\vb b_a \cdot \vb H\sups{inc}}
  \Big\}
\\
 F_i\sups{ext} &=
 \frac{1}{2\omega}\text{Im }\sum_{a}
  \Big\{  k_a^* \Expval{\vb b_a \cdot \partial_i\vb E\sups{inc}}
         +n_a^* \Expval{\vb b_a \cdot \partial_i \vb H\sups{inc}}
  \Big\}
\\
 \mc T_i\sups{ext} &=
 \frac{1}{2\omega}\text{Im }\sum_{a}
  \bigg\{  k_a^*\Big[ \Expval{\vb b_a \times \vb E\sups{inc}}
                     +\Expval{\vb b_a \cdot \partial_{\theta_i}\vb E\sups{inc}}
               \Big]
\\
&\hspace{1in}
         + n_a^*\Big[ \Expval{\vb b_a \times \vb H\sups{inc}}
                     +\Expval{\vb b_a \cdot \partial_{\theta_i}\vb H\sups{inc}}
               \Big]
  \bigg\}.
\end{align*}
Here I have used the shorthand notation 
$$ \vb b \cdot \partial_i \vb E = \sum_{j=1}^3 b_j \partial_i E_j, 
   \qquad 
   \vb b \cdot \partial_{\theta_i}\vb E = 
   \epsilon_{ijk} \sum_{\ell=1}^3 b_\ell X_j \partial_k E_\ell,
   \qquad \vb X = (\vb x_a - \vb x_0)
$$
where $\vb x_0$ is the origin about which we figure the torque.
The integrals here are non-singular two-dimensional integrals over
the basis-function supports, with the integrand involving values 
and derivatives of the incident fields. These are evaluated 
in {\sc scuff-em} by simple low-order numerical cubature.

\subsection{Scattered EMTPFT}

The components of the scattered fields at a point $\vb x_a$ in 
medium $r$ are
%====================================================================%
\begin{align*}
 E_i\sups{scat}(\vb x_a)
 &= \sum_b \left\{ i\omega \mu k_b \vb e_b(\vb x_a)
                   -n_b \vb h_b(\vb x_a) \right\},
\\
 H_i\sups{scat}(\vb x_a)
 &= \sum_b \left\{  k_b \vb h_b(\vb x_a) 
                   +i\omega \epsilon n_b \vb h_b(\vb x_a) \right\}
\end{align*}
%====================================================================%
where $\epsilon=\epsilon_0\epsilon^r,\mu=\mu_0\mu^r$ are the absolute 
permittivity and permeability of the medium and the 
``reduced fields'' $\{\vb e_b, \vb h_b\}$ of basis function $\vb b_b$ 
have components
%====================================================================%
$$ e_{bi}(\vb x_a) =
   \int \mb G_{ij}(\vb r_{ab})b_{bj}(\vb x_b)\,d\vb x_b, 
  \qquad
   h_{bi}(\vb x_a) =
   -\int \wh{\mb C}_{ij}(\vb r_{ab})b_{bj}(\vb x_b)
$$
%====================================================================%
with
%====================================================================%
$$\vb r_{ab} \equiv \vb x_a - \vb x_b, 
  \qquad
  \mb G_{ij}(\vb r) 
   = \Big[ \delta_{ij} + \frac{1}{k^2}\partial_i\partial_j\Big]\Phi(|\vb r|),
  \qquad
  \wh{\mb C}_{ij}(\vb r) = \varepsilon_{ijk} r_k \Psi(|\vb r|), 
$$
$$ \Phi(r) = \frac{e^{ikr}}{4\pi r}, 
   \qquad 
   \Psi(r) = (ikr-1)\frac{e^{ikr}}{4\pi r^3}.
$$
%====================================================================%
(Note that $\Psi$ is defined such that $\nabla \Phi=\vb r\Psi.$
Also, the hat on $\wh{\mb C}$ differentiates it from the tensor $\mb C$
that I use elsewhere in the {\sc scuff-em} documentation;
they are related by $\wh{\mb C}=ik\mb C.$)

The scattered contributions to PFT quantities take the form
%%====================================================================%
\begin{align*}
Q\sups{scat}
&= \mp\frac{1}{2}
   \int\Big\{
    \vb K^*(\vb x_a) \circledast \vb E\sups{scat}(\vb x_a)
  + \vb N^*(\vb x_a) \circledast \vb H\sups{scat}(\vb x_a)
      \Big\} \, d\vb x
\\
&=\mp \frac{1}{2}\text{Re, Im}
  \sum_{ab} \Big\{  k_a^* k_b Q\supt{KK}_{ab}
                   + n_a^* n_b Q\supt{NN}_{ab}
                   +\Big[k_a^* n_b - n_a^* k_b\Big] Q\supt{KNmNK}_{ab}
            \Big\}.
\end{align*}
%====================================================================%
with
%====================================================================%
\begin{subequations}
\begin{align}
Q\supt{KK}_{ab} 
 &\equiv i\omega \mu \iint \vb b_a \circledast \mb G \star \vb b_b \, d\vb x_a \, d\vb x_b
\\[5pt]
Q\supt{NN}_{ab} 
 &\equiv i\omega \epsilon \iint \vb b_a \circledast \mb G \star \vb b_b \, d\vb x_a \, d\vb x_b
\\[5pt]
Q\supt{KNmNK}_{ab} 
 &\equiv \iint \vb b_a \circledast \wh{\mb C} \star \vb b_b \, d\vb x_a \, d\vb x_b
\\[5pt]
\end{align}
\label{QDef}
\end{subequations}
%%====================================================================%
I refer to the $Q_{ab}$ quantities here as ``EMTPFT integrals.''

\subsubsection{Specific forms of EMTPFT integrals}
For the power, force and torque, $Q=\{P, F_i, \mc T_i\}$,
the EMTPFT integrals take the following specific forms.
(In all cases, the position arguments $\vb x_a$ and $\vb x_b$
to basis functions $\vb b_a$ and $\vb b_b$ are suppressed;
the $\vb x_a$ and $\vb x_b$ integrals run over the
supports of $\vb b_a$ and $\vb b_b$.
Also, NN integrals are related to KK integrals by 
$\mu \leftrightarrow \epsilon.$)

%%====================================================================%
\begin{subequations}
\begin{align}
 P\supt{KK}_{ab}
&=i\omega \mu \int \vb b_a \cdot \vb e_b \, dV 
\nn
&=i\omega \mu 
  \underbrace{\iint b_{ai} \mb G_{ij} b_{bj} \, dV \, dV^\prime}_{\vb G_{ab}}
\\
 P\supt{KNmNK}_{ab}
&=\int \vb b_a \cdot \vb h_b \, dV 
\nn
&=\underbrace{\iint b_{ai} \wh{\mb C}_{ij} b_{bj} \, dV \, dV^\prime}
            _{\wh{\vb C}_{ab}}
\end{align}
\label{PIntegrals}
\end{subequations}
%%====================================================================%

\subsection{Using symmetry to simplify self-contributions to PFT}

In evaluating the contributions of surface currents on a body to the
PFT on that body, we may avail ourselves of useful simplifications
that follow from the reciprocity of the Green's functions, namely
$$ \mb G_{ij}(\vb x_a - \vb x_b) = \mb G_{ji}(\vb x_b-\vb x_a), 
   \qquad
   \mb C_{ij}(\vb x_a - \vb x_b) = \mb C_{ji}(\vb x_b-\vb x_a),
$$
a consequence of which is the fact that the EMTPFT integrals $Q_{ab}$
are either symmetric or antisymmetric ($Q_{ab}=\pm Q_{ba}$). For any
summations in which indices $a$ and $b$ run over the same range 
one finds
%====================================================================%
\begin{align*}
   \text{Re }\sum_{ab} k^*_a k_b Q_{ab} 
&=\begin{cases}
   +\sum_{ab} 
    \Big[\text{Re }k_a^* k_b\Big]\Big[\text{Re }Q_{ab}\Big],
    \qquad\qquad &\text{if } Q_{ab} = +Q_{ba} \\
\\
   -\sum_{ab}
    \Big[\text{Im }k_a^* k_b\Big]\Big[\text{Im }Q_{ab}\Big],
    \qquad\qquad &\text{if } Q_{ab} = -Q_{ba} \\
  \end{cases}
\\[10pt]
   \text{Im }\sum_{ab} k^*_a k_b Q_{ab} 
&=\begin{cases}
    \sum_{ab} 
    \Big[\text{Re }k_a^* k_b\Big]\Big[\text{Im }Q_{ab}\Big],
    \qquad\qquad &\text{if } Q_{ab} = +Q_{ba} \\
\\
    \sum_{ab} 
    \Big[\text{Im }k_a^* k_b\Big]\Big[\text{Re }Q_{ab}\Big],
    \quad\qquad &\text{if } Q_{ab} = -Q_{ba} \\
  \end{cases}
\\[10pt]
   \text{Re }\sum_{ab} (k^*_a n_b-n_a^* k_b) Q_{ab}
&=\begin{cases}
   -\Big[\text{Im }(k_a^* n_b - n_a^* k_b)\Big]\Big[\text{Im }Q_{ab}\Big],
    \qquad &\text{if } Q_{ab} = +Q_{ba} \\
\\
    \sum_{ab}
    \Big[\text{Re }(k_a^* n_b - n_a^* k_b)\Big]\Big[\text{Re }Q_{ab}\Big],
    \qquad &\text{if } Q_{ab} = -Q_{ba} \\
  \end{cases}
\\[10pt]
   \text{Im }\sum_{ab} (k^*_a n_b-n_a^* k_b) Q_{ab}
&=\begin{cases}
    \sum_{ab}
    \Big[\text{Im }(k_a^* n_b - n_a^* k_b)\Big]\Big[\text{Re }Q_{ab}\Big],
    \qquad &\text{if } Q_{ab} = +Q_{ba} \\
\\
    \sum_{ab}
    \Big[\text{Re }(k_a^* n_b - n_a^* k_b)\Big]\Big[\text{Im }Q_{ab}\Big],
    \qquad &\text{if } Q_{ab} = -Q_{ba} \\
  \end{cases}
\end{align*}
%====================================================================%

%%%%%%%%%%%%%%%%%%%%%%%%%%%%%%%%%%%%%%%%%%%%%%%%%%%%%%%%%%%%%%%%%%%%%%
%%%%%%%%%%%%%%%%%%%%%%%%%%%%%%%%%%%%%%%%%%%%%%%%%%%%%%%%%%%%%%%%%%%%%%
%%%%%%%%%%%%%%%%%%%%%%%%%%%%%%%%%%%%%%%%%%%%%%%%%%%%%%%%%%%%%%%%%%%%%%
\subsubsection{Self-contribution to power}

The power EMTPFT integrals are symmetric, i.e.
$P\supt{KK}_{ab}=P\supt{KK}_{ba}$ and similarly for 
$P\supt{NN}_{ab}$ and $P\supt{KNmNK}_{ab}$. Using the 
table of symmetry rules above yields

%====================================================================%
\begin{align}
P\sups{scat}
&=\mp \frac{1}{2}\text{Re }
  \sum_{ab} \Big\{  k_a^* k_b P\supt{KK}_{ab}
                   + n_a^* n_b P\supt{NN}_{ab}
                   +\Big[k_a^* n_b - n_a^* k_b\Big] P\supt{KNmNK}_{ab}
            \Big\}
\\
&=\mp \frac{1}{2}
  \sum_{ab} \Big\{ \Big[\text{Re } k_a^* k_b \Big]
                   \Big[\text{Re } P\supt{KK}_{ab} \Big]
                  +\Big[\text{Re } n_a^* n_b \Big]
                   \Big[\text{Re } P\supt{NN}_{ab} \Big]
\nn
&\hspace{1.5in}
                  -\Big[\text{Im } (k_a^* n_b - n_a^* k_b)\Big]
                   \Big[\text{Im } P\supt{KNmNK}_{ab}
            \Big\}
\nonumber
\end{align}
From (\ref{PIntegrals}) we have 
$$ \text{Re } P_{ab}\supt{KK} 
   = \text{Re }\Big[i\omega \mu \vb G_{ab}\Big]
   = -\omega\text{Im}(\mu\vb G_{ab})
$$
%====================================================================%
so ultimately we find 
%====================================================================%
\begin{align}
 P\sups{scat} 
&= \pm \frac{1}{2}
   \sum_{ab} \Big\{ 
     \omega \mu_0 \Big[\text{Re } k_a^* k_b \Big]
                  \Big[\text{Im }\mu^r \vb G_{ab}\Big]
    +\omega \epsilon_0 \Big[\text{Re } n_a^* n_b \Big]
                  \Big[\text{Im }\epsilon^r \vb G_{ab}\Big]
\nn
&\hspace{1.5in}
    +\Big[\text{Im } (k_a^* n_b - n_a^* k_b)\Big]
     \Big[\text{Im } \wh{\vb C}_{ab}\Big]
  \Big\}.
\label{PScatSymmetry}
\end{align}
%====================================================================%

%%%%%%%%%%%%%%%%%%%%%%%%%%%%%%%%%%%%%%%%%%%%%%%%%%%%%%%%%%%%%%%%%%%%%%
%%%%%%%%%%%%%%%%%%%%%%%%%%%%%%%%%%%%%%%%%%%%%%%%%%%%%%%%%%%%%%%%%%%%%%
%%%%%%%%%%%%%%%%%%%%%%%%%%%%%%%%%%%%%%%%%%%%%%%%%%%%%%%%%%%%%%%%%%%%%%
\subsubsection{Self-contribution to force and torque}

The force EMTPFT integrals are anti-symmetric, i.e.
$F\supt{KK}_{ab}=-F\supt{KK}_{ba}$ and similarly for
$F\supt{NN}_{ab}$ and $F\supt{KNmNK}_{ab}$. Using the
table of symmetry rules above yields
%====================================================================%
\begin{align}
F_i\sups{scat}
&=\mp \frac{1}{2\omega}\text{Im }
  \sum_{ab} \Big\{  k_a^* k_b F\supt{KK}_{ab}
                   + n_a^* n_b F\supt{NN}_{ab}
                   +\Big[k_a^* n_b - n_a^* k_b\Big] F\supt{KNmNK}_{ab}
            \Big\}
\\
&=\mp \frac{1}{2\omega}
  \sum_{ab} \Big\{ \Big[\text{Im } k_a^* k_b \Big]
                   \Big[\text{Re } F\supt{KK}_{ab} \Big]
                  +\Big[\text{Im } n_a^* n_b \Big]
                   \Big[\text{Re } F\supt{NN}_{ab} \Big]
\nn
&\hspace{1.5in}
                  +\Big[\text{Re } (k_a^* n_b - n_a^* k_b)\Big]
                   \Big[\text{Im } F\supt{KNmNK}_{ab}\Big]
            \Big\}
\nonumber
\end{align}
%====================================================================%
From (\ref{FIntegrals}) we have 
$$ \text{Re } F_{ab}\supt{KK} 
   = \text{Re }\Big[i\omega \mu \partial_i \vb G_{ab}\Big]
   = -\omega\text{Im}(\mu \partial_i \vb G_{ab})
$$
%====================================================================%
so ultimately we find 
%====================================================================%
\begin{align}
 F_i\sups{scat} 
&= \pm \frac{1}{2}
   \sum_{ab} \Big\{ 
     \mu_0 \Big[\text{Im } k_a^* k_b \Big]
                  \Big[\text{Im }\mu^r \partial_i \vb G_{ab}\Big]
    +\epsilon_0 \Big[\text{Im } n_a^* n_b \Big]
                  \Big[\text{Im }\epsilon^r \partial_i \vb G_{ab}\Big]
\nn
&\hspace{1.5in}
    -\frac{1}{\omega} 
     \Big[\text{Re } (k_a^* n_b - n_a^* k_b)\Big]
            \Big[\text{Im } \partial_i \wh{\vb C}_{ab}\Big]
  \Big\}.
\label{FScatSymmetry}
\end{align}
%====================================================================%
