\documentclass[letterpaper]{article}
\usepackage[square,sort,comma,numbers]{natbib}
\usepackage{array}
%====================================================================%
/home/homer/work/scuff-em/tex/scufftex.tex
\graphicspath{{figures/}}
\renewcommand{\wt}{\widetilde}

%------------------------------------------------------------
%------------------------------------------------------------
%- Special commands for this document -----------------------
%------------------------------------------------------------
%------------------------------------------------------------

%------------------------------------------------------------
%------------------------------------------------------------
%- Document header  -----------------------------------------
%------------------------------------------------------------
%------------------------------------------------------------
\title {Electrostatic Green's function above dielectric substrate
        and ground plane 
       }
\author {Homer Reid}
\date {March 2, 2017}

%------------------------------------------------------------
%------------------------------------------------------------
%- Start of actual document
%------------------------------------------------------------
%------------------------------------------------------------

\begin{document}
\pagestyle{myheadings}
\markright{Homer Reid: Electrostatic GF above PCB} 
\maketitle

%\tableofcontents

I will construct an expression for the potential
at $\vb x=(x,y,z)$ due to a unit-strength point charge at
$\vb x^\prime=(x^\prime,y^\prime,z^\prime)$ in the presence
of a dielectric substrate of thickness $T$ and relative
permittivity $\epsilon_r$ above a ground plane at $z=0$.
This potential consists of three contributions:
%====================================================================%
$$ \bmc G(T, \epsilon_r; \vbrho, z, z^\prime)
   = G\sups{vacuum} + G\sups{image} + G\sups{surface+image}
$$
%====================================================================%
Here $\vbrho\equiv (x-x^\prime)\vbhat{x} + (y-y^\prime)\vbhat{y}$
is the tranverse separation and the various contributions are
%====================================================================%
\begin{enumerate}
 \item $G\sups{vacuum}$ is the direct contribution
       of the point charge,
       $$G\sups{vacuum}(\vbrho, z, z^\prime)
         =G_0(\vbrho,z,z^\prime)\equiv
           \frac{1}{4\pi\sqrt{|\vbrho|^2 + (z-z^\prime)^2}}.
       $$
 \item $G\sups{image}$ is the contribution of the image charge
       due to the ground plane,
       $$G\sups{image}(\vbrho, z, z^\prime)
         =-G_0(\vbrho,z,-z^\prime).
       $$
 \item $G\sups{surface+image}$ is the contribution of an
       induced surface-charge layer $\sigma(\vb x)$ living 
       at the upper surface of the dielectric substrate ($z=T$),
       together with the corresponding image-charge layer 
       $-\sigma$ at $z=-T$.
\end{enumerate}
%====================================================================%
Proceeding in the spirit of integral-equation methods,
my strategy will be first to determine $\sigma(\vb x)$ to satisfy
the boundary conditions at the dielectric interface, then 
compute $\mc G$ as an surface integral involving $\sigma.$

\section{Potential of surface charge layer}

I first write down an expression for the potential due to
the infinite surface-charge layer.

\subsection*{2D Fourier representation of free-space Green's function}

The 3D Fourier representation of the free-space Green's function---the
potential at $\vb x=(\vbrho,z)$ due to a point charge at 
$\vb x^\prime=(0,z^\prime)$---is 
%====================================================================%
\begin{align}
G_0(\vbrho, z, z^\prime)
&=
 \frac{1}{4\pi |\vb x-\vb x^\prime|}
\nn
 &=\int \frac{d^3 \vb k}{(2\pi)^3}
        \frac{e^{i\vb k \cdot (\vb x-\vb x^\prime)}}{|\vb k|^2}
\nonumber
\intertext{Evaluate the $k_z$ integral:}
 &=\frac{1}{2}\int \frac{d^2 \vb q}{(2\pi)^2}
        \frac{e^{i\vb q\cdot \vbrho-q|z-z^\prime|}}{q},
 \qquad  \vb   q\equiv  \vb k_\parallel,
 \quad        q\equiv |\vb q|.
\label{G02D}
\end{align}
%====================================================================%

\subsection*{Potential due to surface-charge layer at $z=T$}

Using (\ref{G02D}), 
the potential above and below a surface-charge layer at $z=T$
(not yet accounting for the ground plane) reads
%====================================================================%
\begin{align}
 G\sups{surface}(\vbrho, z)
   &=\int \, G_0(\vbrho-\vb y,z, T)\, \sigma(\vb y) \, d^2 \vb y
\nn
   &=\frac{1}{2} \int \frac{d^2 \vb q}{(2\pi)^2}
     \frac{e^{i\vb q \cdot \vbrho - q|z-T|}}{q}
     \underbrace{\int d\vb y \, \sigma(\vb y) \, e^{-i\vb q \cdot \vb y}}
               _{\wt{\sigma}(\vb q)}
\nn
   &=\int \frac{d^2 \vb q}{(2\pi)^2}
     \frac{\wt\sigma(\vb q)e^{i\vb q \cdot \vbrho - q|z-T|}}{2q}
\label{GSurface}
\end{align}
%====================================================================%
where $\wt \sigma$ is the Fourier transform of $\sigma$.
For circularly-symmetric cases in which $\wt \sigma$ depends only
on the magnitude of $\vb q$, as will be the case here,
this can be simplified to read
%====================================================================%
$$ G\sups{surface}(\vbrho, z)
   =\int \frac{dq}{4\pi} \wt \sigma(q) J_0(q\rho) e^{-q|z-T|}
$$
with $\rho=|\vbrho|$.
%====================================================================%

\subsection*{Potential due to surface-charge layer in presence of 
             ground plane}

In the presence of the ground plane at $z=0$, equation (\ref{GSurface})
is augmented by the contribution of the image-charge
layer $-\sigma$ at $z=-T$:
%====================================================================%
\begin{align}
G\sups{surface+image}(\vbrho,z)
 &=\int \frac{dq}{2\pi} \wt \sigma(q) J_0(q\rho)
 \begin{cases} e^{-qz}\sinh qT, \qquad &\hphantom{0<}\,\,z>T \\
               e^{-qT}\sinh qz, \qquad &0<z<T.
 \end{cases}
\label{GSurfaceImage}
\end{align}
%====================================================================%


\subsection{Full expressions for potential and $z-$directed electric
            field above and inside dielectric substrate}

Combining all contributions, the 
potential and its $z$-derivative at points above the dielectric layer $(z>T)$
read
%====================================================================%
\begin{align}
&\hspace{-0.3in}G\sups{above}(\vbrho, z, z^\prime)
\nn
  &= G_0(\vbrho, z, z^\prime) - G_0(\vbrho, z, -z^\prime)
    + \int \frac{dq}{2\pi} \wt\sigma(q) J_0(q\rho) e^{-qz} \sinh qT
\label{GAbove}\\
&\hspace{-0.3in} -\partial_z G\sups{above}(\vbrho, z, z^\prime)
\nn
  &= E_{z0}(\vbrho, z, z^\prime) - E_{z0}(\vbrho, z, -z^\prime)
    + \int \frac{q\,dq}{2\pi} \wt\sigma(q) J_0(q\rho) e^{-qz} \sinh qT
\label{dGdzAbove}
\end{align}
where I put
$$ E_{z0}(\vbrho, z, z^\prime)
   =\frac{(z-z^\prime)}{4\pi\big[|\vbrho|^2 + (z-z^\prime)^2\big]^{3/2}}.
$$
%====================================================================%
Inside the dielectric layer $(0<z<T)$ we have
%====================================================================%
\begin{align}
&\hspace{-0.3in}G\sups{inside}(\vbrho, z, z^\prime)
\nn
  &= G_0(\vbrho, z, z^\prime) - G_0(\vbrho, z, -z^\prime)
    + \int \frac{dq}{2\pi} \wt\sigma(q)J_0(q\rho) e^{-qT}\sinh qz
\label{GInside} \\
&\hspace{-0.3in} -\partial_z G\sups{inside}(\vbrho, z, z^\prime)
\nn
  &= E_{z0}(\vbrho, z, z^\prime) - E_{z0}(\vbrho, z, -z^\prime)
    - \int \frac{q\,dq}{2\pi}\wt\sigma(q)J_0(q\rho) e^{-qT}\cosh qz
\label{dGdzInside}
\end{align}
%====================================================================%
The boundary condition at $z=T$ is 
%====================================================================%
$$            \left|\pard{G\sups{above}}{z}\right|_{z=T} = 
   \epsilon_r \left|\pard{G\sups{inside}}{z}\right|_{z=T} 
$$
%====================================================================%
Insert (\ref{dGdzAbove}) and (\ref{dGdzInside}):
%====================================================================%
\begin{align*}
&\hspace{-0.2in}
\int \frac{q\,dq}{2\pi} 
  \wt \sigma(q) J_0(q\rho) e^{-qT}\Big[\sinh qT + \epsilon_r \cosh qT\Big]
&=(\epsilon_r-1)
  \Big[ E_{z0}(\vbrho,T,z^\prime)-E_{z0}(\vbrho,T,-z^\prime)\Big]
\end{align*}
or, inverse Bessel-transforming\footnote{Multiply both 
sides by $\rho J_0(q^\prime\rho)$, integrate over all $\rho$, and
use 
$$\int_0^\infty \rho J_0(q\rho) J_0(q^\prime \rho) d\rho = 
  \frac{1}{q}\delta(q-q^\prime).$$}.
%====================================================================%
\begin{align}
\wt \sigma(q)
&=2\pi \frac{\epsilon_r-1}{\Upsilon(\epsilon_r, T, q)}
  \int \rho J_0(q\rho)
  \Big[   E_{z0}(\vbrho, T, z^\prime)
        - E_{z0}(\vbrho, T, -z^\prime)
  \Big] \, d\rho 
\label{Sigma1}
\end{align}
%====================================================================%
where I put
%====================================================================%
$$ \Upsilon(\epsilon_r, T, q) \equiv
   e^{-qT}\Big[\sinh qT + \epsilon_r \cosh qT\Big].
$$
%====================================================================%
The integral we need here is
%====================================================================%
$$ \int_0^\infty \frac{ u \Delta  J_0(qu)}{(u^2 + \Delta^2)^{3/2}}\,du
   =e^{-q\Delta}
$$
%====================================================================%
Then (\ref{Sigma1}) reads
%====================================================================%
\begin{align*}
 \wt{\sigma}(q)
 &=\frac{\epsilon_r-1}{2\Upsilon(\epsilon_r,T,q)}
   \Big[ e^{-q(T-z^\prime)} - e^{-q(T+z^\prime)}\Big]
\\
 \wt{\sigma}(q)
 &=\frac{\epsilon_r-1}{\sinh qT + \epsilon_r \cosh qT}
   \sinh qz^\prime.
\end{align*}
%====================================================================%
Inserting back into (\ref{GAbove}-\ref{dGdzInside}), the final 
Green's function reads
%====================================================================%
\begin{align}
 G(\vbrho, z, z^\prime)
  &= G_0(\vbrho, z, z^\prime) - G_0(\vbrho, z, -z^\prime)
\nn
&\quad 
    +\int \frac{dq}{2\pi} 
          \frac{\epsilon_r-1}{\sinh qT + \epsilon_r \cosh qT} J_0(q\rho)
          \sinh qz^\prime
          \begin{cases}
            e^{-qz} \sinh qT, \qquad &\hphantom{0<}z>T \\
            e^{-qT} \sinh qz, \qquad &0<z<T
          \end{cases}
\nonumber
\end{align}
%====================================================================%

\end{document}
