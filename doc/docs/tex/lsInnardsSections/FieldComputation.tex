%%%%%%%%%%%%%%%%%%%%%%%%%%%%%%%%%%%%%%%%%%%%%%%%%%%%%%%%%%%%%%%%%%%%%%
%%%%%%%%%%%%%%%%%%%%%%%%%%%%%%%%%%%%%%%%%%%%%%%%%%%%%%%%%%%%%%%%%%%%%%
%%%%%%%%%%%%%%%%%%%%%%%%%%%%%%%%%%%%%%%%%%%%%%%%%%%%%%%%%%%%%%%%%%%%%%
\newpage
\section{Computation of the fields in \lss geometries}

Using the fields of individual basis functions as
discussed in the previous section, we can compute the 
total $\vb E$ and $\vb H$ fields at arbitrary points 
in a {\sc scuff-em} geometry.


\subsection{The simplest case}

The simplest case to consider is that in which we have
a single compact body $\mc B$ in vacuum; let the body surface
be $\mc S=\partial \mc B$ and suppose the incident fields 
arise from sources lying outside $\mc B$. Let the surface-current
coefficients be $\{\vb k_\alpha, \vb n_\alpha\}$, so that 
the electric and magnetic surface currents are
%====================================================================%
$$ \vb K(\vb x) = \sum_{\alpha} k_\alpha \vb b_\alpha(\vb x),
   \qquad
   \vb N(\vb x) = \sum_{\alpha} n_\alpha \vb b_\alpha(\vb x)
$$
%====================================================================%
where $\{\vb b_\alpha\}$ is the set of RWG basis functions 
on surface $\partial \mc B$. Then the fields at a point outside 
$\mc B$ are
%====================================================================%
\begin{subequations}
\begin{align}
 \vb E(\vb x) 
  &= \vb E\sups{inc}(\vb x) + 
\sum_{\alpha} \Big\{ ik_0Z_0 k_\alpha \vb e_\alpha(k_0; \vb x)
                             -n_\alpha \vb h_\alpha(k_0; \vb x)
                   \Big\}
\\
 \vb H(\vb x) 
  &= \vb H\sups{inc}(\vb x) + 
     \sum_{\alpha} \Big\{     k_\alpha \vb h_\alpha(k_0; \vb x)
                         +\frac{ik_0}{Z_0} n_\alpha \vb e_\alpha(k_0; \vb x)
                   \Big\}.
\end{align}
\label{OutsideFields}
\end{subequations}
%====================================================================%
The fields at a point inside $\mc B$ are 
%====================================================================%
\begin{subequations}
\begin{align}
 \vb E(\vb x)
  &= -\sum_{\alpha} \Big\{ ik_1Z_1 k_\alpha \vb e_\alpha(k_1; \vb x)
                                  -n_\alpha \vb h_\alpha(k_1; \vb x)
                   \Big\}
\\
 \vb H(\vb x) 
  &= -\sum_{\alpha} \Big\{     k_\alpha \vb h_\alpha(k_1; \vb x)
                         +\frac{ik_1}{Z_1} n_\alpha \vb e_\alpha(k_1; \vb x)
                   \Big\}.
\end{align}
\label{InsideFields}
\end{subequations}
%====================================================================%
Note the following differences between (\ref{OutsideFields}) and
(\ref{InsideFields}):
\begin{itemize}
 \item The incident fields contribute to (\ref{OutsideFields})
       but not to (\ref{InsideFields}).
 \item Equation (\ref{InsideFields}) involves a minus sign that 
       is not present in (\ref{OutsideFields}).
 \item Equation (\ref{OutsideFields}) involves the vacuum
       wavenumber $k_0=\omega/c$ and the vacuum 
       wave impedance $Z_0\approx 377 \, \Omega.$
       Equation (\ref{InsideFields}) involves the wavenumber 
       $k_1=\sqrt{\epsilon^r \mu^r}k_0$ and wave impedance 
       $Z_1=\sqrt{\frac{\mu^r}{\epsilon^r}}Z_0$ for the body interior.
       (Here $\{\epsilon^r, \mu^r\}$ are the relative permittivity 
        and permeability of the medium inside body $\mc B$ at the 
        frequency in question.)
\end{itemize}

\subsection{The general case}

The previous discussion was for the simplest case of a 
single compact body in vacuum with incident-field 
sources lying outside the body. The generalization
to more complicated cases is straighforward. 

Consider a point $\vb x$ in some region $\mathcal{R}$
of a \lss geometry. In general, $\mathcal{R}$
will be bounded by some collection of surfaces
$\{\mathcal{S}^s\}$. Let's subdivide the set of 
surfaces bounding $\mathcal{R}$ into two groups:
a first set $\{\mathcal{S}^\alpha\}$ for which
$\mathcal{R}$ is the \textit{exterior} surface,
and a second set $\{\mathcal{S}^\beta\}$ for which
$\mathcal{R}$ is the \textit{interior} surface.
Then the electric field at $\vb x$ is
%====================================================================%
\begin{align} 
E_i(\vb x) 
    = &\hphantom{-} \sum_\alpha \int_{\mathcal{S}^\alpha} 
           \Big\{ \Gamma_{ij}\supt{EE}(\mathcal{R}; \vb x, \vb x^\prime) 
                   K_j(\vb x^\prime)
                  \,+\,
                   \Gamma_{ij}\supt{EM}(\mathcal{R}; \vb x, \vb x^\prime) 
                   N_j(\vb x^\prime)
           \Big\} \, d\vb x^\prime
\nn
     &-\sum_\beta \int_{\mathcal{S}^\beta} 
           \Big\{ \Gamma_{ij}\supt{EE}(\mathcal{R}; \vb x, \vb x^\prime) 
                   K_j(\vb x^\prime)
                  \,+\,
                   \Gamma_{ij}\supt{EM}(\mathcal{R}; \vb x, \vb x^\prime) 
                   N_j(\vb x^\prime)
           \Big\} \, d\vb x^\prime
\nn
     &+E^{\text{\scriptsize inc},r}_i(\vb x).
\label{FieldContributionsEquation}
\end{align}
%====================================================================%
In this equation, $\vb E\incr$ is the field due to any 
incident field sources that lie inside the region $\mathcal{R}^r.$
(The $\vb H$ fields are given by identical relations with 
$\{ \Gamma\supt{EE}, \Gamma\supt{EM}, E\incr\} \to 
 \{ \Gamma\supt{ME}, \Gamma\supt{MM}, H\incr\}$.)

Note the following points with respect to equation
(\ref{FieldContributionsEquation}):
\begin{itemize}
  \item Sources on surfaces $\mathcal{S}_\alpha$ for which
        $\mathcal{R}$ is the exterior medium  
        contribute to the fields at $\vb x$ 
        with a positive sign.
        Sources on surfaces $\mathcal{S}_\beta$ for which
        $\mathcal{R}$ is the interior medium  
        contribute to the fields at $\vb x$ 
        with a negative sign.

  \item In each line of (\ref{FieldContributionsEquation}), 
        (i.e. regardless of the surface over which we are 
        integrating) the Green's functions used to compute
        the fields at $\vb x$ are the Green's functions for
        the homogeneous region $\mathcal{R}$ containing $\vb x$.
        The material properties of other regions are not
        referenced.
\end{itemize}

%####################################################################%
%####################################################################%
%####################################################################%
\begin{figure}
\begin{center}
%\includegraphics{FieldContributionsFigure}
\caption{Contributions of objects to the scattered fields at 
         an arbitrary point $\vb x$. Objects $\mathcal{O}_\beta$
         and $\mathcal{O}_\gamma$ contribute
         to the field at $\vb x$ ``with a plus sign'' 
         (cf. equation \ref{FieldContributionsEquation}).
         Object $\mathcal{O}_\alpha$ 
         contributes to the field at $\vb x$ ``with a minus
         sign.'' Objects $\mathcal{O}_\delta$ and $\mathcal{O}_\lambda$
         do not contribute to the field at $\vb x$.}
\label{FieldContributionsFigure}
\end{center}
\end{figure}
